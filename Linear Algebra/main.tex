\documentclass[12pt,openany]{book}

%%%%%%%%%%%%%%%%%%%%%%%%%%%%% Using Packages %%%%%%%%%%%%%%%%%%%%%%%%%%%%%%%%%%
\usepackage[letterpaper, margin=1 in]{geometry}
\usepackage{
    parskip,
    amsmath,
    amssymb,
    nicematrix,
    amsthm,
    enumitem,
    ulem,
    hyperref,
    subfiles
}
\usepackage[english]{babel}
\usepackage[autostyle]{csquotes}

\hypersetup{
    colorlinks,
    citecolor=black,
    filecolor=black,
    linkcolor=black,
    urlcolor=black
}

%%%%%%%%%%%%%%%%%%%%%%%%%%%%%%%%%%%%%%%%%%%%%%%%%%%%%%%%%%%%%%%%%%%%%%%%%%%%%%%

\newtheorem{theorem}{Theorem}

\newtheoremstyle{mydefinitionstyle}{\topsep}{\topsep}{}{}{\itshape}{:}{\newline}{}
\theoremstyle{mydefinitionstyle}
\newtheorem*{definition}{Definition}

\newtheoremstyle{myexamplestyle}{\topsep}{\topsep}{}{}{\itshape}{:}{\newline}{}
\theoremstyle{myexamplestyle}
\newtheorem*{example}{Example}

\theoremstyle{remark}
\newtheorem*{note}{Note}
\newtheorem*{aside}{Aside}
\newtheorem{corollary}{Corollary}[theorem]

\renewcommand\qedsymbol{$\square$}
\newcommand{\msout}[1]{	ext{\sout{\ensuremath{#1}}}}

% Other Settings

\DeclareMathOperator{\ima}{Im}
\DeclareMathOperator{\rref}{rref}

%%%%%%%%%%%%%%%%%%%%%%%%%%%%%%% Title & Author %%%%%%%%%%%%%%%%%%%%%%%%%%%%%%%%
\title{Linear Algebra\thanks{Partially based on courses taught by Dr. Eric Brussel and Dr. Morgan Sherman at California Polytechnic State University San Luis Obispo}}
\author{Christopher K. Walsh}
%%%%%%%%%%%%%%%%%%%%%%%%%%%%%%%%%%%%%%%%%%%%%%%%%%%%%%%%%%%%%%%%%%%%%%%%%%%%%%%

\begin{document}

\maketitle
\tableofcontents

\newpage
\chapter*{Preface}
\addcontentsline{toc}{section}{Preface}
This document compiles all my notes on linear algebra including any self-study.
This also serves as a foundation for the other analysis topics covered in my notes series.

This topic is what really propelled me to study more advanced math. Linear algebra is a lot more than the study of
vectors and systems of equations. It is a methodology for vector spaces including standard Euclidean vectors and functions.
For the first time in my life, I started dealing with math beyond calculations and computations. I had to prove things
about these systems and I found that incredibly interesting and rewarding.

Hope that these notes make you feel the same way.

- Christopher

%%%%%%%%%%%%%%%%%%%%%%%%%%%%%%%%%%%%%%%%%%%%%%%%%%%%%%%%%%%%%%%%%%%%%%%%%%%%%%%%%%%%%%%%%%%%%%%%%%%%%%%%%%%%%%%

\chapter{Matricies}
\subfile{sections/matricies.tex}

\chapter{Application: Systems of Linear Equations}
\subfile{sections/linear_eqn.tex}

\chapter{Vectors and Vector Spaces}
\subfile{sections/vector_spaces.tex}

\chapter{Linear Transformations}
\subfile{sections/linear_transform.tex}

\chapter{Eigenvalue Problems}
\subfile{sections/eigenvalue_problems.tex}

\end{document}