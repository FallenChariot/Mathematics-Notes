\documentclass[12pt,openany]{book}

%%%%%%%%%%%%%%%%%%%%%%%%%%%%% Using Packages %%%%%%%%%%%%%%%%%%%%%%%%%%%%%%%%%%
\usepackage[letterpaper, margin=1 in]{geometry}
\usepackage{parskip}
\usepackage{amsmath}
\usepackage{amssymb}
\usepackage{amsthm}
\usepackage{enumitem}
\usepackage{ulem}
\usepackage{hyperref}
\usepackage{subfiles}

\hypersetup{
    colorlinks,
    citecolor=black,
    filecolor=black,
    linkcolor=black,
    urlcolor=black
}

%%%%%%%%%%%%%%%%%%%%%%%%%%%%%%%%%%%%%%%%%%%%%%%%%%%%%%%%%%%%%%%%%%%%%%%%%%%%%%%

\newtheorem{theorem}{Theorem}

\newtheoremstyle{mydefinitionstyle}{\topsep}{\topsep}{\itshape}{}{\itshape}{:}{\newline}{}
\theoremstyle{mydefinitionstyle}
\newtheorem*{definition}{Definition}

\newtheoremstyle{myexamplestyle}{\topsep}{\topsep}{}{}{\itshape}{:}{\newline}{}
\theoremstyle{myexamplestyle}
\newtheorem*{example}{Example}

\theoremstyle{remark}
\newtheorem*{note}{Note}
\newtheorem*{aside}{Aside}

\renewcommand\qedsymbol{$\square$}
\newcommand{\msout}[1]{	ext{\sout{\ensuremath{#1}}}}

% Other Settings

%%%%%%%%%%%%%%%%%%%%%%%%%%%%%%% Title & Author %%%%%%%%%%%%%%%%%%%%%%%%%%%%%%%%
\title{Linear Algebra\thanks{Partially based on courses taught by Dr. Eric Brussel and Dr. Morgan Sherman at California Polytechnic State University San Luis Obispo}}
\author{Christopher K. Walsh}
%%%%%%%%%%%%%%%%%%%%%%%%%%%%%%%%%%%%%%%%%%%%%%%%%%%%%%%%%%%%%%%%%%%%%%%%%%%%%%%

\begin{document}

\maketitle
\tableofcontents

\newpage
\section*{Preface}
\addcontentsline{toc}{section}{Preface}
This document compiles all my notes on linear algebra including any self-study (e.g. advanced applications in modern physics).
This also serves as a foundation for the other analysis topics covered in my notes series.

For most engineers, linear algebra is simply the study of matrices and vectors in \( \mathbb{R}^n \).
For a mathematician, linear algebra is the study of vector spaces and linear transformations.
This definition means linear algebra is more abstract than vectors describing physical quantities or systems of linear equations.
They can be functions for example, and thinking of functions in a vector space is surprisingly useful for advanced differential equations.
In modern or quantum physics, all the behaviors of particles can be modeled using complex vector spaces (see sections on Hilbert spaces and complex vector spaces).
Thus, there is a strong motivation in science to understand linear algebra hollistically.

If you need this for school and industry, I hope that you are able to do whatever you are trying to accomplish.
If you are here because this interests you, I hope you find this as entertaining as I did.

- Christopher

%%%%%%%%%%%%%%%%%%%%%%%%%%%%%%%%%%%%%%%%%%%%%%%%%%%%%%%%%%%%%%%%%%%%%%%%%%%%%%%%%%%%%%%%%%%%%%%%%%%%%%%%%%%%%%%

\chapter{Matricies}
\subfile{sections/matricies.tex}

\end{document}