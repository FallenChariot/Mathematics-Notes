\documentclass[../main.tex]{subfiles}

\begin{document}

\section{Introduction to Linear Transformations}

Let \( A \) be an \( m \times n \) matrix. We can interpret it as a function or transformation between vector spaces,
where \( A: \mathbb{R}^n \rightarrow \mathbb{R}^m \).

Note that \( A \) is a linear transformation since \( A(\vec{v}+\vec{w}) = A\vec{v} + A\vec{w} \)
and \( A(c\vec{v}) = cA\vec{v} \).

\begin{example}[]
    \[ A = \begin{bmatrix}
        \frac{1}{\sqrt{2}} & \frac{-1}{\sqrt{2}} \\
        \frac{1}{\sqrt{2}} & \frac{1}{\sqrt{2}}
    \end{bmatrix} \]

    What does it do?

    Let \[ \vec{e_1} = \begin{bmatrix} 0 \\ 1 \end{bmatrix}, \vec{e_2} = \begin{bmatrix} 1 \\ 0 \end{bmatrix} \].
    % insert diagram
\end{example}

\begin{example}
    Find a matrix for \( \frac{\Pi}{2} \) rotation.
    \[ A =\begin{bmatrix}
        0 & 1 \\
        1 & 0
    \end{bmatrix} \]
\end{example}

%%%%%%%%%%%%%%%%%%%%%%%%%%%%%%%%%%%%%%%%%%%%%%%%%%%%%%%%%%%%%%%%%%%%%%%%%%%%%%%%%%%%%%%%%%%%%%%%%%%%%%%%%%%%%%

\section{Eigenvalues and Eigenvectors}

Most linear transformations can be understood with eigenvalues and eigenvectors.

\begin{definition}[]
    Let \( A \) be \( n \times n \).

    An \textbf{eigenvalue} of \( A \) is a scalar \( \lambda \) such that \( A\vec{v}=\lambda\vec{v} \)
    has a nonzero solution \(\vec{v}\).

    An \textbf{eigenvector} \( \vec{v} \) for \( \lambda \) is a nonzero \(\vec{v}\): \( A\vec{v}=\lambda\vec{v} \).

    An \textbf{eigenspace} for \( \lambda \) is the set of all \( \vec{v} \): \( A\vec{v}=\lambda\vec{v} \).

    An \textbf{eigenbasis} for \( \lambda \) is a basis \( \lambda \)'s eigenspace.

\end{definition}

We will look at a simple matrix to give concrete examples for all of these definitions.

\begin{example}[]
    Given \( A \; n \times n \), find its eigenvalues, eigenvectors, eigenspace, and eigenbasis.

    There are two eigenvalues, \( \lambda = 2, 1 \).

    For \( \lambda =2 \), \( \vec{e_1} \) is one possible eigenvector. Another possible eigenvector
    is \( 3\vec{e_1} \). The eigenspace for \( \lambda=2 \) is the x-axis. The eigenbasis is simply
    \( \{\vec{e_1}\} \).

    For \( \lambda=1 \), the eigenbasis is simply \( \{\vec{e_2}\} \).

    We can determine this intutitively by considering some vectors and applying the linear transformation \( A \).

    % insert figure

\end{example}

To determine the eigenvalues and eigenvectors analytically, note that \( A\vec{v}=\lambda\vec{v} \) for nonzero \( \vec{v} \)
is the same as \( (A-\lambda I)\vec{v}=\vec{0} \). Thus all \( \lambda \) satisfy \( \det(A-\lambda I) =0 \).

\end{document}