\documentclass[../main.tex]{subfiles}

\begin{document}

%%%%%%%%%%%%%%%%%%%%%%%%%%%%%%%%%%%%%%%%%%%%%%%%%%%%%%%%%%%%%%%%%%%%%%%%%%%%%%%%%%%%%%%%%%%%%%%%%%%%%%%%%%%%%%%

\section{Defining Matrices}

Let \( n,\, m \) be two integers \( \geq 1 \).
A \textbf{matrix} is an array of numbers with \( m \) rows and \( n \) columns (called a \( m \times n \) matrix).

We call \textbf{\( a_{ij} \)} the \textbf{ij-entry} which is the entry in the \textit{i}th row and the \textit{j}th column.
We write a matrix often as \( A=(a_{ij}) \) and define \( a_{ij} \).

Each column of an \( m \times n \) matrix is a \textbf{column vector}.
Each row of an \( m \times n \) matrix is a \textbf{row vector}.

\begin{example}[Identity Matrix]
    The Kronecker delta is defined as follows:
    \[
        \delta_{ij} =
        \begin{cases}
            1,& i = j \\
            2,& i \neq j
        \end{cases}
    \]

    Then we define the \textbf{identity matrix} as:
    \[ \mathbf{I_{n \times n}} = (\delta_{ij}) =
        \begin{bmatrix}
            1 & 0 &  \dots  & 0 \\
            0 & 1 & \dots & 0 \\
            \vdots & \vdots & \ddots & \vdots \\
            0 & 0 & \dots  & 1
        \end{bmatrix}
    \]
\end{example}

\begin{example}[]
    If we have a matrix, \( A = \begin{bmatrix}
        3 & 4 \\
        1 & -1 \\
        2 & 2 
    \end{bmatrix}, \)
    the second column vector of \( A \) is \( \begin{bmatrix}
        4 \\
        1 \\
        2
    \end{bmatrix} \)
    and the second row vector of \( A \) is \( \begin{bmatrix}
        1 & -1
    \end{bmatrix} \).

    We can describe matrices using their column or row vectors. For example:
    \[ A = \begin{bmatrix}
        \vec{r} \\
        \vec{s} \\
        \vec{t}
    \end{bmatrix} \]
    where \[ \vec{r} = \begin{bmatrix}
        3 & 4
    \end{bmatrix}, \, \vec{s} = \begin{bmatrix}
        1 & -1
    \end{bmatrix}, \, \vec{t} = \begin{bmatrix}
        2 & 2
    \end{bmatrix} \]

    Or: \[ A = \begin{bmatrix}
        \vec{a} & \vec{b}
    \end{bmatrix} \]
    where \[ \vec{a} = \begin{bmatrix}
        3 \\
        1 \\
        2
    \end{bmatrix}, \, \vec{b} = \begin{bmatrix}
        4 \\
        -1 \\
        2
    \end{bmatrix} \]
\end{example}


We treat matricies the same way as numbers.
Let \( A \) be an \( m \times n \) matrix and \( B \) be an \( p \times q \) matrix.
\begin{itemize}[mode=unboxed]
    \item We can add \( A \) and \( B \). If \( m=p \) and \( n=q \), then \( A+B = ( a_{ij}+b_{ij} ) \)
    \item We can multiply by a scalar \( c \): \( c \cdot A = ( c \cdot a_{ij} ) \)
    \item We can multiply \( A \) and \( B \). If \( n=p \), then \( A \cdot B = (c_{ij}) \) where \( c_{ij} = \vec{r_{i}}(A) \cdot \vec{c_{i}}(B) \).
        \emph{Caution:} In general, \( A \cdot B \neq B \cdot A \).
    \item We also define the transpose of a matrix. The transpose of \( A \) is \( A^t = (d_{ij}) \) where \( d_{ij} = a_{ji} \).
        When we take a transpose, we switch the columns into rows and vice versa.
\end{itemize}

Certain special matrices can be described with other terminology.
Suppose we have a matrix, \( A = (a_{ij}), i = 1, \dots ,m \, \text{and} \, j = 1, \dots ,n \).
\begin{itemize}
    \item If \( m=n \), then A is a \textbf{square matrix}.
    \item If \( A^t = A \), then A is a \textbf{symmetric matrix}. Note: this means A must also be square.
    \item If \( A^t = -A \), then A is said to be \textbf{skew-symmetric}.
    \item If for all \( i,j \) such that \( i \neq j \), \( a_{ij} = 0 \), then A is called \textbf{diagonal}.
\end{itemize}

%%%%%%%%%%%%%%%%%%%%%%%%%%%%%%%%%%%%%%%%%%%%%%%%%%%%%%%%%%%%%%%%%%%%%%%%%%%%%%%%%%%%%%%%%%%%%%%%%%%%%%%%%%%%%%%

\section{Matricies as Linear Transformations}

Matricies can be used to model transformations of vectors from \( \mathbb{R}^n \) to \( \mathbb{R}^m \).
This is accomplished by having an \( m \times n \) matrix, \( A \), written as: \[ A: \mathbb{R}^n \rightarrow \mathbb{R}^m \]

\begin{example}[\( \mathbb{R}^n \) to \( \mathbb{R}^m \)]
    
    Let \( A = \begin{bmatrix}
        3 & 4 \\
        1 & -1 \\
        2 & 2 
    \end{bmatrix}, \) and
    \( \vec{v} = \begin{bmatrix}
        2 \\
        1
    \end{bmatrix} \).
    Thus, \( A: \mathbb{R}^2 \) to \( \mathbb{R}^3 \) and:
    \[
        A\vec{v} = \begin{bmatrix}
            3 & 4 \\
            1 & -1 \\
            2 & 2 
        \end{bmatrix}
        %
        \begin{bmatrix}
            2 \\
            1
        \end{bmatrix}
        =
        \begin{bmatrix}
            10 \\
            1 \\
            6
        \end{bmatrix}
    \]

    \begin{note}
        We are getting a linear combination of the column vectors of A.
        In other words, \( A \vec{v} = x \vec{a}+ y \vec{b} \).
    \end{note}
\end{example}

\end{document}