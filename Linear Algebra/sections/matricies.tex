\documentclass[../main.tex]{subfiles}

\begin{document}

%%%%%%%%%%%%%%%%%%%%%%%%%%%%%%%%%%%%%%%%%%%%%%%%%%%%%%%%%%%%%%%%%%%%%%%%%%%%%%%%%%%%%%%%%%%%%%%%%%%%%%%%%%%%%%%

\section{Defining Matrices}

Let \( n,\, m \) be two integers \( \geq 1 \).
A \textbf{matrix} is an array of numbers with \( m \) rows and \( n \) columns (called a \( m \times n \) matrix).

We call \textbf{\( a_{ij} \)} the \textbf{ij-entry} which is the entry in the \textit{i}th row and the \textit{j}th column.
We write a matrix often as \( A=(a_{ij}) \) and define \( a_{ij} \).

Each column of an \( m \times n \) matrix is a \textbf{column vector}.
Each row of an \( m \times n \) matrix is a \textbf{row vector}.

\begin{example}[Identity Matrix]
    The Kronecker delta is defined as follows:
    \[
        \delta_{ij} =
        \begin{cases}
            1,& i = j \\
            2,& i \neq j
        \end{cases}
    \]

    Then we define the \textbf{identity matrix} as:
    \[ I_n = (\delta_{ij}) =
        \begin{bmatrix}
            1 & 0 &  \dots  & 0 \\
            0 & 1 & \dots & 0 \\
            \vdots & \vdots & \ddots & \vdots \\
            0 & 0 & \dots  & 1
        \end{bmatrix}
    \]
\end{example}

\begin{example}[]
    If we have a matrix, \( A = \begin{bmatrix}
        3 & 4 \\
        1 & -1 \\
        2 & 2 
    \end{bmatrix}, \)
    the second column vector of \( A \) is \( \begin{bmatrix}
        4 \\
        1 \\
        2
    \end{bmatrix} \)
    and the second row vector of \( A \) is \( \begin{bmatrix}
        1 & -1
    \end{bmatrix} \).

    We can describe matrices using their column or row vectors. For example:
    \[ A = \begin{bmatrix}
        \vec{r} \\
        \vec{s} \\
        \vec{t}
    \end{bmatrix} \]
    where \[ \vec{r} = \begin{bmatrix}
        3 & 4
    \end{bmatrix}, \, \vec{s} = \begin{bmatrix}
        1 & -1
    \end{bmatrix}, \, \vec{t} = \begin{bmatrix}
        2 & 2
    \end{bmatrix} \]

    Or: \[ A = \begin{bmatrix}
        \vec{a} & \vec{b}
    \end{bmatrix} \]
    where \[ \vec{a} = \begin{bmatrix}
        3 \\
        1 \\
        2
    \end{bmatrix}, \, \vec{b} = \begin{bmatrix}
        4 \\
        -1 \\
        2
    \end{bmatrix} \]
\end{example}

%%%%%%%%%%%%%%%%%%%%%%%%%%%%%%%%%%%%%%%%%%%%%%%%%%%%%%%%%%%%%%%%%%%%%%%%%%%%%%%%%%%%%%%%%%%%%%%%%%%%%%%%%%%%%%%

\subsection{Matrix Operations}

We treat matricies the same way as numbers.
Let \( A \) be an \( m \times n \) matrix and \( B \) be an \( p \times q \) matrix.
\begin{itemize}[mode=unboxed]
    \item We can add \( A \) and \( B \). If \( m=p \) and \( n=q \), then \( A+B = ( a_{ij}+b_{ij} ) \)
    \item We can multiply by a scalar \( c \): \( c \cdot A = ( c \cdot a_{ij} ) \)
    \item We can multiply \( A \) and \( B \). If \( n=p \), then \( A \cdot B = (c_{ij}) \) where \( c_{ij} = \vec{r_{i}}(A) \cdot \vec{c_{i}}(B) \).
        \emph{Caution:} In general, \( A \cdot B \neq B \cdot A \).
    \item We also define the transpose of a matrix. The transpose of \( A \) is \( A^t = (d_{ij}) \) where \( d_{ij} = a_{ji} \).
        When we take a transpose, we switch the columns into rows and vice versa.
\end{itemize}

Certain special matrices can be described with other terminology.
Suppose we have a matrix, \( A = (a_{ij}), i = 1, \dots ,m \, \text{and} \, j = 1, \dots ,n \).
\begin{itemize}
    \item If \( m=n \), then A is a \textbf{square matrix}.
    \item If \( A^t = A \), then A is a \textbf{symmetric matrix}. Note: this means A must also be square.
    \item If \( A^t = -A \), then A is said to be \textbf{skew-symmetric}.
    \item If for all \( i,j \) such that \( i \neq j \), \( a_{ij} = 0 \), then A is called \textbf{diagonal}.
\end{itemize}

\subsubsection{Determinants}

The determinant is a property of a square matrix, \( A \).

\underline{Geomtric Def'n.}

Let \( A = \begin{bmatrix}
    \vec{c_1}, & \vec{c_2}, & \dots, & \vec{c_n}
\end{bmatrix} \)
(vectors in \( \mathbb{R}^n \)).
Let \( \Pi = \) parallelotope defined by basing them all at the same point.
Then \( V^\sigma(\Pi) = \text{det} \begin{bmatrix}
    \vec{c_1}, & \vec{c_2}, & \dots, & \vec{c_n}
\end{bmatrix} \) (signed \( n \) volume).

\begin{example}[]
    \[ A = \begin{bmatrix}
        \vec{c_1} & \vec{c_2}
    \end{bmatrix} = \begin{bmatrix}
        0 & 1 \\
        1 & 0
    \end{bmatrix} \]
    \[ \text{det}A = -1 \]

    % ADD DIAGRAM
\end{example}

\underline{Algebraic Def'n.}

Let \( A = \left( a_{ij} \right), \, n \times n\).
Let \( A_{ij} = \) submatrix obtained from A by eliminating row \( i \) and column \( j, \, (n-1) \times (n-1) \) matrix.

The minor of \( A \) is \( M_{ij} = \text{det}A_{ij}\).
The cofactor of \( A \) is \( C_{ij} = (-1)^{i+j}\text{det}A_{ij}\).

Then \( \text{det}A = \sum_{j=1}^{n}a_{ij}C_{ij} \) (for any \( i, \, 1 \leq i \leq n \)).
We call this the cofactor expansion along the \( i \)th row.

\begin{example}[]
    \[ A = \begin{bmatrix}
        7 & 3 & 12 \\
        2 & 5 & 8 \\
        1 & 5 & 2
    \end{bmatrix} \]
    Find the determinant.

    Use row 3.
    \[ \text{det}A = \sum_{j=1}^{3} a_{3j} C_{3j} = a_{31}C_{31} + a_{32}C_{32} + a_{33}C_{33} \]

    
\end{example}

%%%%%%%%%%%%%%%%%%%%%%%%%%%%%%%%%%%%%%%%%%%%%%%%%%%%%%%%%%%%%%%%%%%%%%%%%%%%%%%%%%%%%%%%%%%%%%%%%%%%%%%%%%%%%%%

\subsection{Inverse Matricies}

An \( n \times n \) matrix \( A \) may or may not have an \textbf{inverse}: A matrix \( B \) such that
\[ AB=BA = \mathbf{I}_n = \begin{bmatrix}
    1 & 0 &  \dots  & 0 \\
    0 & 1 & \dots & 0 \\
    \vdots & \vdots & \ddots & \vdots \\
    0 & 0 & \dots  & 1
\end{bmatrix}
\]

We write \( B = A^{-1} \).

For a linear system, \( A \vec{x} = \vec{b} \) with \( A, \; n \times n, \) if \( A \) is invertible:
\begin{gather*}
    \underbrace{A^{-1}A}_{I_n} \vec{x} = A^{-1} \vec{b} \\
    \therefore \, \text{Sol'n. is } \vec{x} = A^{-1} \vec{b}
\end{gather*}

We can vary \( \vec{b} \) and the solutions are immediate.

For the \( 2 \times 2 \) case,
\( A = \begin{bmatrix}
    a & b \\
    c & d
\end{bmatrix} \)

\( A \) has an inverse \( \iff |A| = ad - bc \neq 0 \).

Then \[ A^-1 = \frac{1}{ad-bc}
    \begin{bmatrix}
        d & -b \\
        -c & a
    \end{bmatrix} 
\]

\begin{example}[]
    \[ A = \begin{bmatrix}
        2 & 5 \\
        7 & 9
    \end{bmatrix} \]
    Invertible?

    \( 18 - 35 = -17 \neq 0 \) \checkmark
    \[ A^{-1} = \frac{-1}{17} \begin{bmatrix}
        9 & -5 \\
        -7 & 2
    \end{bmatrix} \]

    \underline{Check}
    \begin{gather*}
        \frac{-1}{17}
        \begin{bmatrix}
            9 & -5 \\
            -7 & 2
        \end{bmatrix}
        \begin{bmatrix}
            2 & 5 \\
            7 & 9
        \end{bmatrix}
        = \frac{-1}{17}
        \begin{bmatrix}
            -17 & 0 \\
            0 & -17
        \end{bmatrix}
        = \begin{bmatrix}
            1 & 0 \\
            0 & 1
        \end{bmatrix}
    \end{gather*}
\end{example}

\begin{theorem}[]
    A \( n \times n \) matrix \( A \) has an inverse \( \iff \text{rk}A = n \).
    If \( A \) has an inverse, then \( A^{-1} \) is given by \( \text{rref}\left[ A \, I_n \right] = \left[ I_n \, A^-1 \right] \)
\end{theorem}

\begin{example}[]
    \[ A = \begin{bmatrix}
        1 & 2 & -1 \\
        2 & 5 & -1 \\
        1 & 2 & 0
    \end{bmatrix} \]
    Is it invertible?
    \begin{align*}
        &\begin{bmatrix}
            1 & 2 & -1 & 1 & 0 & 0 \\
            2 & 5 & -1 & 0 & 1 & 0 \\
            1 & 2 & 0 & 0 & 0 & 1
        \end{bmatrix} \\[2mm]
        %
        \xrightarrow[2. \, A_{13}(-1)]{1. \, A_{12}(-2)}
        &\begin{bmatrix}
            1 & 2 & -1 & 1 & 0 & 0 \\
            0 & 1 & 1 & -2 & 1 & 0 \\
            0 & 0 & 1 & -1 & 0 & 1
        \end{bmatrix} \\[2mm]
        %
        \xrightarrow[4. \, A_{31}(1)]{3. \, A_{32}(-1)}
        &\begin{bmatrix}
            1 & 2 & 0 & 0 & 0 & 1 \\
            0 & 1 & 0 & -1 & 1 & -1 \\
            0 & 0 & 1 & -1 & 0 & 1
        \end{bmatrix} \\[2mm]
        %
        \xrightarrow{5. \, A_{21}(-2)}
        &\begin{bmatrix}
            1 & 0 & 0 & 2 & -2 & 3 \\
            0 & 1 & 0 & -1 & 1 & -1 \\
            0 & 0 & 1 & -1 & 0 & 1
        \end{bmatrix}
    \end{align*}
    \[ A^{-1} = \begin{bmatrix}
        2 & -2 & 3 \\
        -1 & 1 & -1 \\
        -1 & 0 & 1
    \end{bmatrix} \]

    \[ AA^{-1} =
        \begin{bmatrix}
            1 & 2 & -1 \\
            2 & 5 & -1 \\
            1 & 2 & 0
        \end{bmatrix}
        \begin{bmatrix}
            2 & -2 & 3 \\
            -1 & 1 & -1 \\
            -1 & 0 & 1
        \end{bmatrix}
        = \begin{bmatrix}
            1 & 0 & 0 \\
            0 & 1 & 0 \\
            0 & 0 & 1
        \end{bmatrix}
    \]
\end{example}

\underline{Properties of \( A^{-1} \)}
\begin{enumerate}
    \item If \( A \) is invertible, so is \( A^{-1}, \, \left( A^{-1} \right)^{-1} = A \).
    \item If \( A, \, B \) are invertible, \( n \times n \), then so is \( A \cdot B \) and \( (BA)^{-1} = B^{-1}A^{-1} \).

        Witness \( A \left( B \cdot B^-1 \right)A^{-1} = A I_n A^-1 = I_n \).
    \item If \( A \) is invertible, so is \( A^t, \, \left( A^t \right)^{-1} = \left( A^{-1} \right)^t \).
\end{enumerate}

\begin{theorem}[]
    Suppose \( A \) is \( n \times n \).
    Then the following are equivalent
    \begin{enumerate}
        \item \( A \) is invertible
        \item \( A \vec{x} = \vec{b} \) has a unique sol'n, \( \forall \vec{b} \)
        \item \( \text{rk}(A) = n \)
        \item \( \text{rref}(A) = I_n \)
        \item \( \text{det}(A) \neq 0 \)
    \end{enumerate}
\end{theorem}

%%%%%%%%%%%%%%%%%%%%%%%%%%%%%%%%%%%%%%%%%%%%%%%%%%%%%%%%%%%%%%%%%%%%%%%%%%%%%%%%%%%%%%%%%%%%%%%%%%%%%%%%%%%%%%%

\section{Matricies as Linear Transformations}

Matricies can be used to model transformations of vectors from \( \mathbb{R}^n \) to \( \mathbb{R}^m \).
This is accomplished by having an \( m \times n \) matrix, \( A \), written as: \[ A: \mathbb{R}^n \rightarrow \mathbb{R}^m \]

\begin{example}[\( \mathbb{R}^n \) to \( \mathbb{R}^m \)]
    
    Let \( A = \begin{bmatrix}
        3 & 4 \\
        1 & -1 \\
        2 & 2 
    \end{bmatrix}, \) and
    \( \vec{v} = \begin{bmatrix}
        2 \\
        1
    \end{bmatrix} \).
    Thus, \( A: \mathbb{R}^2 \) to \( \mathbb{R}^3 \) and:
    \[
        A\vec{v} = \begin{bmatrix}
            3 & 4 \\
            1 & -1 \\
            2 & 2 
        \end{bmatrix}
        %
        \begin{bmatrix}
            2 \\
            1
        \end{bmatrix}
        =
        \begin{bmatrix}
            10 \\
            1 \\
            6
        \end{bmatrix}
    \]

    \begin{note}
        We are getting a linear combination of the column vectors of A.
        In other words, \( A \vec{v} = x \vec{a}+ y \vec{b} \).
    \end{note}
\end{example}

\end{document}