\documentclass[../main.tex]{subfiles}

\begin{document}

\section{Determinants and Cross Products}

\subsection{Determinants}

\begin{example}[Computing areas]
    Suppose we have a paralellogram made by the vectors \( (1,4) \text{ and }  (8,3) \). Compute the area.

    % insert figure

    We could compute the area of the paralellogram using geometry, but we can easily see the area as 29 using the determinant.
    \[ A = \left| \det \begin{bmatrix}
        1 & 8 \\
        4 & 3
    \end{bmatrix}\right| = \left|1 \cdot 3 - 8 \cdot 4\right| = \left|-29\right| = 29 \]
\end{example}

We can interpret this matrix as a linear transformation from the standard basis in \( \mathbb{R}^2 \), a change of basis.

\[ e_1 = \begin{bmatrix}
    0 \\ 1
\end{bmatrix} \mapsto f_1 = \begin{bmatrix}
    1 \\ 4
\end{bmatrix} \]

\[ e_2 = \begin{bmatrix}
    1 \\ 0
\end{bmatrix} \mapsto f_2 = \begin{bmatrix}
    8 \\ 3
\end{bmatrix} \]

\subsection{Cross Product}
With the standard basis in \( \mathbb{R}^3 \), let \( u = (u_1, u_2, u_3) = u_1e_1 + u_2e_2 + u_3e_3 \)
and \( v = (v_1, v_2, v_3) = v_1e_1 + v_2e_2 + v_3e_3 \).

We define the cross product of \( u \wedge  v \) as:
\[ u \wedge  v = \left( \begin{bmatrix}
    u_2 & u_3 \\
    v_2 & v_3
\end{bmatrix}, \;  -\begin{bmatrix}
    u_1 & u_3 \\
    v_1 & v_3
\end{bmatrix}, \; \begin{bmatrix}
    u_1 & u_2 \\
    v_1 & v_2
\end{bmatrix} \right) \]

This also has a geometric interpretation: the three components of the cross product are the areas of the "shadows"
of the the paralellogram formed by \( u \text{ and } v \) in \( yz, \; xz, \; xy \) -planes respectively.

\underline{Properties:}
\begin{enumerate}
    \item The area of the paralellogram \( u,v \) is equal to \( \left| u \wedge v \right| \). 
    \item The direction of \( u \wedge v \) (if non-zero) is perpendicular to the paralellogram \( u,v \) using the right hand rule.
    \item The volume of the paralellopiped \( u,v,w \) is equal to \( (u \wedge v) \cdot w = \det \begin{bmatrix}
        u & v & w
    \end{bmatrix} = \det \begin{bmatrix}
        u \\ v \\ w
    \end{bmatrix} \)
    \item \( u \wedge v \) =  \( - v \wedge u \)
    \item \( u \wedge (v+w) = u \wedge v + u \wedge w \)
    \item \( \left( u(t) \wedge v(t) \right)' = u'(t) \wedge v(t) + u(t) \wedge v'(t) \)
\end{enumerate}

\end{document}