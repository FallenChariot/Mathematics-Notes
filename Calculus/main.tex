%%%%%%%%%%%%%%%%%%%%%%%%%%%%% Define Document Class %%%%%%%%%%%%%%%%%%%%%%%%%%%%%%%%%%
\documentclass[12pt,openany]{book}
%%%%%%%%%%%%%%%%%%%%%%%%%%%%%%%%%%%%%%%%%%%%%%%%%%%%%%%%%%%%%%%%%%%%%%%%%%%%%%%

%%%%%%%%%%%%%%%%%%%%%%%%%%%%% Using Packages %%%%%%%%%%%%%%%%%%%%%%%%%%%%%%%%%%
\usepackage{graphicx}
\usepackage{parskip}
\usepackage{amsmath}
\usepackage{amssymb}
\usepackage{amsthm}
\usepackage{enumitem}
\usepackage{ulem}
\usepackage{hyperref}
\usepackage{subfiles}

\hypersetup{
    colorlinks,
    citecolor=black,
    filecolor=black,
    linkcolor=black,
    urlcolor=black
}

\graphicspath{{images/}}

%%%%%%%%%%%%%%%%%%%%%%%%%%%%%%%%%%%%%%%%%%%%%%%%%%%%%%%%%%%%%%%%%%%%%%%%%%%%%%%

\newtheorem{theorem}{Theorem}

\newtheoremstyle{mydefinitionstyle}{\topsep}{\topsep}{\itshape}{}{\itshape}{:}{\newline}{}
\theoremstyle{mydefinitionstyle}
\newtheorem*{definition}{Definition}

\newtheoremstyle{myexamplestyle}{\topsep}{\topsep}{}{}{\itshape}{:}{\newline}{}
\theoremstyle{myexamplestyle}
\newtheorem*{example}{Example}

\theoremstyle{remark}
\newtheorem*{note}{Note}
\newtheorem*{aside}{Aside}

\renewcommand\qedsymbol{$\square$}
\newcommand{\msout}[1]{	ext{\sout{\ensuremath{#1}}}}

% Other Settings

%%%%%%%%%%%%%%%%%%%%%%%%%%%%%%% Title & Author %%%%%%%%%%%%%%%%%%%%%%%%%%%%%%%%
\title{Title}
\author{Christopher K. Walsh}
%%%%%%%%%%%%%%%%%%%%%%%%%%%%%%%%%%%%%%%%%%%%%%%%%%%%%%%%%%%%%%%%%%%%%%%%%%%%%%%

\begin{document}

\maketitle
\tableofcontents

\newpage
\section*{Preface}
\addcontentsline{toc}{section}{Preface}
This document compiles all my notes on differential equations including any self-study.
Some sections require content from other math subjects to be completed understood (such as linear algebra or other analysis topics).
The corresponding topic and section will be referred to in-text.

If you need this for school and industry, I hope that you are able to do whatever you are trying to accomplish.
If you are here because this interests you, I hope you find this as entertaining as I did.

- Christopher

\chapter{Sequences and Series}
\subfile{sections/infinite_series.tex}

\end{document}