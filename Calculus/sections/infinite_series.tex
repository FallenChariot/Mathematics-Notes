\documentclass[../main.tex]{subfiles}

\begin{document}

%%%%%%%%%%%%%%%%%%%%%%%%%%%%%%%%%%%%%%%%%%%%%%%%%%%%%%%%%%%%%%%%%%%%%%%%%%%%%%%%%%%%%%%%%%%%%%%%%%%%%%%%%%%%%%%

\begin{definition}[Infinite Series]
    An infinite series is an expression where we add an infinite number of elements together:
    \[ \sum_{n=1}^{\infty}{a_{k}}=a_{1}+a_{2}+a_{3}+\dots \]
    \( a_{k} \) 
\end{definition}

\begin{definition}[Partial Sum]
    We write the nth partial sum:
    \[ S_n = \sum_{k=1}^n a_{k} = a_{1} + a_{2} + a_{3} + \dots +a_{n} \]
\end{definition}

We can define a sequence {\( S_{n} \)}. We say that:
\begin{itemize}
    \item \( \sum_{k=1}^\infty a_{k} \) converges if {\( S_{n} \)} converges.
    \item \( \sum_{k=1}^\infty a_{k} \) diverges to \( \pm \infty \) if {\( S_{n} \)} diverges to \( \pm \infty \).
    \item \( \sum_{k=1}^\infty a_{k} \) really diverges if {\( S_{n} \)} really diverges.
\end{itemize}

\begin{example}[Geometric Series (r=1/2)]
    The following series converges to 1.
    \[ \sum_{k=1}^\infty \frac{1}{2^k}= 1/2 + 1/4 + 1/8 + \dots \]
    This is a geometric series since \( \frac{a_{k+1}}{a_{k}} \) equals some constant (independent of k). Here:
    \[ \frac{\frac{1}{2^{k+1}}}{\frac{1}{2^k}}=\frac{1}{2} \]
\end{example}

\end{document}