\documentclass[../main.tex]{subfiles}

\begin{document}

\section{Sequences}

The concepts of sequences arise naturally and recur throughout mathematics (e.g. analysis).

\begin{definition}[Sequences]
    A \textbf{sequence} of numbers is a (possibly infinite) list with ordered elements:
    \[ \displaystyle \{a_n\}^{\infty}_{n=1} = \{a_1, a_2, a_3, \dots \}\]
    (We could start the sequence at \( x=0 \) if we prefer.)
\end{definition}

Some examples include:
\begin{enumerate}
    \item \( \{1, 2, 3, 4, \dots \} \quad a_n = n \) \\
        This reproduces the natural numbers.
        We can also define this sequence recursively with: \( a_{n+1} = a_n +1 \; (a_1 = 1) \).
    \item \( \{1, 3, 6, 10, \dots \} \quad a_n = a_n + (n+1), \; a_1 =1  \) \\
        These are the triangular numbers. A closed formula for them is \( a_n = \frac{1}{2}n(n+1) \).
    \item \( \{2, 4, 7, 11, 16, \dots \} \quad a_{n+1} = a_n +(n+1), \; a_1 = 2 \)
    \item \( \{1,1,2,3,5,8,13,21, \dots \} \) \\
        This is the famous Fibonacci sequence.
\end{enumerate}

\textit{Aside:} The Fibonacci sequence is arguably the most famous sequence appearing in \href{https://www.fq.math.ca/}{a variety of contexts}.
Its closed formula is: \[ a_n = \frac{\phi^n - \psi^n}{\phi-\psi} \]

where \( \displaystyle \phi = \frac{1+\sqrt(5)}{2} \text{ and } \psi = \frac{1-\sqrt{5}}{2} \).

Numerically, \( \phi \text{ and } \psi \) are roots of the polynomial \( x^2 - x - 1 \).
Geometrically, they are related to the ratios of a regular pentagon:

% add picture of ratios

In this chapter, we will build upon this definition into tools we can use to solve other problems.

%%%%%%%%%%%%%%%%%%%%%%%%%%%%%%%%%%%%%%%%%%%%%%%%%%%%%%%%%%%%%%%%%%%%%%%%%%%%%%%%%%%%%%%%%%%%%%%%%%%%%%%%%%%%%%%

\section{Sequence Convergence}

\begin{definition}[Convergence]
    An infinite sequence \( {a_n} \) \textbf{converges} to a number \( L \) if 
    \( \lim_{n \to \infty} a_n \) exists and equals \( L \).
\end{definition}

Meaning, for any \( \varepsilon > 0 \), there exists a number \( N \) such that \( | a_n - L | < \varepsilon \)
whenever \( n > N \). (This goes back to the definition of limits.)

% Add picture of epsilon corridor

We say \( a_n \) diverges to \( +\infty \) if \( \lim_{n \to \infty} a_n = \infty \)
and \( a_n \) diverges to \( -\infty \) if \( \lim_{n \to \infty} a_n = -\infty \).
The sequence \( a_n \) 'really diverges' otherwise.

\begin{example}[]
    Prove using the definition that \( \frac{1}{n} \) converges to 0.

    Given any \( \varepsilon > 0  \), choose \( N = \lceil{ \frac{1}{\varepsilon} \rceil} \).
    Then, \( n > N = \lceil{ \frac{1}{\varepsilon} \rceil} \implies | \frac{1}{n} - 0 | < \varepsilon \)
    Therefore, by the definition, \( \frac{1}{n} \rightarrow 0 \).
\end{example}

From this definition, we can prove several properties of sequences identical to the Limit Laws.
If \( {a_n} \text{ and } {b_n} \) are convergent sequences and \( c \) is a constant, then:
\[ \lim_{n \to \infty} (a_n \pm b_n) = \lim_{n \to \infty} a_n \pm \lim_{n \to \infty} b_n \]
\[ \lim_{n \to \infty} ca_n = c \lim_{n \to \infty} a_n \]
\[ \lim_{n \to \infty} c = c \]
\[ \lim_{n \to \infty} (a_n b_n) = \lim_{n \to \infty} a_n \cdot \lim_{n \to \infty} b_n \]
\[ \lim_{n \to \infty} \frac{a_n}{b_n} = \frac{\lim_{n \to \infty} a_n}{\lim_{n \to \infty} b_n}
    \text{ if } \lim_{n \to \infty} b_n \neq 0 \]
\[ \lim_{n \to \infty} =  \left[\lim_{n \to \infty} a_n \right]^p \text{ if } p>0 \text{ and } a_n>0\]

\begin{example}[]
    Evaluate \( \{ \frac{3+5n^2}{n+n^2} \} \).

    Dividing the numerator and denominator by \( \frac{1}{n^2} \), we can use the Limit Laws.
    \begin{align*}
        \lim_{n \to \infty} \frac{3+5n^2}{n+n^2} &= \lim_{n \to \infty} \frac{3+5n^2}{n+n^2} \cdot \frac{\frac{1}{n^2}}{\frac{1}{n^2}} \\
        &= \lim_{n \to \infty} \frac{\frac{3}{n^2}+5}{\frac{1}{n}+1} \\
        &= \frac{\lim_{n \to \infty} \frac{3}{n^2} + \lim_{n \to \infty} 5 }{\lim_{n \to \infty} \frac{1}{n} + \lim_{n \to \infty} 1} \\
        &= \frac{0+5}{0+1} = 5
    \end{align*}
\end{example}

Note: To show a sequence \( \{ a_j \} \) diverges, given any \( R>0 \)
find \( N \) such that \( n > N  \) implies \( a_n > R \).

\begin{example}[]
    Show \( \{\ln n\} \to \infty \).

    For a general \( R \), choose \( N = e^R \).
    Then \( n > e^R \implies \ln n > \ln e^R = R \).
    Therefore \( \{ \ln n \} \to \infty \).
\end{example}

When a sequence has an interdeterminant form, it is useful to rewrite the expression in another way to determine its convergence or divergence.
As shown before, we can rewrite polynomial expressions. Another tool we can use is the following theorems (taken from Stewart).
The first theorem allows us to compare sequences with discrete variables to functions of continuous variables when limits approach infinity. 
\begin{theorem}
    \label{stewart3}
    If \( \lim_{x \to \infty} f(x) = L  \) and \( f(n) = a_n \) when \( n \in \mathbb{Z} \), then \( \lim_{n \to \infty} f(n) = L \).
\end{theorem}
The second theorem says that if we apply a continuous function to terms of a convergent sequence, the result is also convergent.
\begin{theorem}
    \label{stewart7}
    If \( \lim_{n \to \infty} = L \) and the function \( f \) is continuous at \( L \), then \( \lim_{n \to \infty} f(a_n) = L \).
\end{theorem}

\begin{example}[]
    Find \( \lim_{n \to \infty} (1+\frac{1}{n})^n \).

    We can apply Thm~\ref{stewart3} and Thm~\ref{stewart7} using \( f(x) = e^x \).
    Then, \( \lim_{n \to \infty} (1+\frac{1}{n})^n = \lim_{x \to \infty} (1+\frac{1}{x})^x \). Now,
    \[ \left( 1+\frac{1}{x} \right)^x = e^{\ln \left( 1+\frac{1}{x} \right)^x } = e^{x \ln \left( 1+\frac{1}{x} \right)} \]
    So, \[ \lim_{x \to \infty} \left(1+\frac{1}{x}\right)^x = e^{ \lim_{x \to \infty} x \ln \left( 1+\frac{1}{x} \right)} \]
    However, \[ \lim_{x \to \infty} x \ln \left( 1+\frac{1}{x} \right) = \lim_{x \to \infty} \frac{\ln \left( 1+\frac{1}{x} \right)}{\frac{1}{x}} \to \frac{0}{0} \]
    Since this is an interdeterminant form, applying L'Hopital's Rule,
    \[ \lim_{x \to \infty} \frac{ \ln\left(1+ \frac{1}{x}\right) }{ \frac{1}{x} } = \lim_{x \to \infty} \frac{\frac{1}{1+\frac{1}{x}}\frac{-1}{x^2}}{\frac{-1}{x^2}}
        = \lim_{x \to \infty} \frac{1}{1+\frac{1}{x}} = 1\]
    Thus, we conclude \( \lim_{n \to \infty} \left( 1 + \frac{1}{n}\right)^n = e^1 = e \).
\end{example}

\subsection{Little-O Notation}

It is useful to describe how quickly a sequence converges compared to some other sequence. The relative "speed" can be compared with Little-Oh notation.

\begin{definition}[Little-Oh Notation]
    Let \( {a_n} \text{ and } {b_n} \) be sequences. We say \( a_n \) is \textbf{little-oh} of \( b_n \) and write \( a_n = o(b_n) \text{ if } \lim_{n \to \infty} \frac{a_n}{b_n} = 0 \).

    We also say \( b_n \) is much faster than \( a_n \) and write \( a_n << b_n \text{ for } n >> 0 \).
\end{definition}

In computer programming, we want algorithms that require less space and time to process inputs. We can compare algorithmns with similar notation.

\begin{example}[]
    Compare \( {n} \text{ and } {\frac{1}{2}n} \).

    Clearly \( \frac{1}{2}n < n \), so \( \frac{1}{2}n < n \). However,
    \[ \lim_{n \to \infty } \frac{\frac{1}{2}n}{n} = \frac{1}{2} \neq 0 \]
    It is not slow enough.
\end{example}

\begin{example}[]
    Show \( \ln(\ln n) = o(\ln n) \).

    \[ \lim_{n \to \infty}\frac{\ln(\ln n)}{\ln n} \to \frac{\infty}{\infty}\]
    An indeterminant form.

    By Thm~\ref{stewart3}, \[ \lim_{n \to \infty}\frac{\ln(\ln n)}{\ln n} = \lim_{x \to \infty} \frac{\ln(\ln x)}{\ln x} \].

    Using L'Hopital's Rule, \[ \lim_{x \to \infty} \frac{\ln(\ln x)}{\ln x} = \lim_{x \to \infty} \frac{\frac{1}{x\ln x}}{\frac{1}{x}}
        = \lim_{x \to \infty} \frac{1}{\ln x} = 0 \]
    
        We conclude \( ln(\ln n) = o(\ln n) \).
\end{example}

\begin{example}[]
    Compare \( {n!} \text{ and } {e^n} \).

    Claim: \( \lim_{n \to \infty} \frac{e^n}{n!} = 0 \)

    Now, consider the first several terms:
    \[ \frac{e^n}{n!} = \frac{e \cdot e \cdot \ldots \cdot e}{1 \cdot 2 \cdot 3 \cdot \ldots \cdot n} \rightarrow \]
    \begin{align*}
        \frac{e}{1}, \; \frac{e \cdot e}{1 \cdot 2}, \; & \frac{e \cdot e \cdot e}{1 \cdot 2 \cdot 3}, \; \frac{e \cdot e \cdot e \cdot e}{1 \cdot 2 \cdot 3 \cdot 4}, \; \dots \\
        > & \frac{e}{3}\cdot\frac{e^2}{2} > \frac{e^2}{12} \cdot \frac{e^2}{2}
    \end{align*}
    We see starting with the third term, they look similar to a geometric series since \( e < 3 \). (Note \( \frac{e}{3} < 1 \text{ and } \frac{e}{4}<1 \).)
    Then,
    \[ 0 < \frac{e^n}{n!} < \frac{e^2}{2}\cdot\frac{e}{n} = \frac{e^3}{2}\left(\frac{1}{n}\right) \]
    By the Squeeze Theorem, \( \frac{e^n}{n!} \to 0 \).
\end{example}

\subsection{Monotonic Bounded Sequence Theorem}

When dealing with recursive sequences, the Monotonic Bounded Sequence Thm (MST) is very helpful in determining convergence.

\begin{theorem}[Monotonic Bounded Sequence Theorem]
    \label{mst}
    Every monotonic, bounded sequence is convergent.
\end{theorem}

We would simply need to show the sequence is bounded and monotonic. If it converges, then the limit is a fixed point where \( a_{n+1} = a_n \).

% insert diagram

\begin{example}[]
    Consider \( a_1 = \sqrt{2}, \, a_n = \sqrt{2+a_n} \).

    \emph{Step 1:} Find the fixed point. (Finding it first makes showing it bounded easier.)
    \begin{align*}
        \sqrt{2+a_n} & = a_n \\
        2 + a_n & = a_n^2 \\
        a_n^2 - a_n -2 & = 0 \\
        (a_n-2)(a_n+1) & = 0
    \end{align*}
    \[ \therefore a_n = 2, -1 \]
    If \( \{a_n\} \) converges, \( \lim_{n \to \infty} a_n = 2 \) (since \( a_1 > 0 \text{ and } a_n \) is increasing).

    \emph{Step 2:} Show it is bounded. Use a fixed point if possible. \\
    Lower bound: -1 (all terms are greater than 0) \\
    Upper bound: 2 \dots \\
    Now, \( a_1 = \sqrt{2} < 2 \)
    If \( a_n < 2 \), then \( a_{n+1} = \sqrt{2+a_n} < \sqrt{2+2} = 2 \), so \( a_n<2 \implies a_{n+1}<2 \).
    Therefore, by the principal of induction, \( a_n<2 \) for all n, and we have shown \( a_n \) is bounded. \checkmark

    \emph{Step 3:} Show it is increasing i.e. \( a_{n+1} > a_n \) for all n.
    Now, this is equivalent to \( a_{n+1}^2 > a_n^2 \). Rearranging and applying \( a_{n+1} = \sqrt{2+a_n} \), we see that \( 2+a_n-a_n^2 > 0 \).
    This expression holds since \( -1 < a_n < 2 \) for all \( n \), so \( \{a_n\} \) is increasing.

    Therefore, by MST, \( \{a_n\} \) converges, so \( \lim_{n \to \infty} a_n = 2 \).
\end{example}

Now, consider this example of a continued fraction:
\[ \displaystyle 1 + \cfrac{1}{1+\cfrac{1}{1+\cfrac{1}{1+\dots}}} \]

This generates the following sequence:
\[ \left\{ 1+1, \, 1+ \frac{1}{1+1}, \, 1+\frac{1}{1+\frac{1}{1+1}}, \, \dots \right\} \]
or equivalently:
\[ \left\{ 2, \, \frac{3}{2}, \, \frac{5}{3}, \, \frac{8}{5}, \, \frac{13}{8}, \, \dots \right\} \]

The sequence is certainly bounded between 1 and 2, but it is not monotonic. However, we can still use Thm~\ref{mst}.

Both \( \{a_{2n}\} \text{ and } \{a_{2n+1}\} \) are monotonic and approach the same limit (which is the Golden Mean \( \phi \)).

\end{document}