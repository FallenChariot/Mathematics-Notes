\documentclass[../main.tex]{subfiles}

\begin{document}

%%%%%%%%%%%%%%%%%%%%%%%%%%%%%%%%%%%%%%%%%%%%%%%%%%%%%%%%%%%%%%%%%%%%%%%%%%%%%%%%%%%%%%%%%%%%%%%%%%%%%%%%%%%%%%%

\section{Defining Partial Derivatives}

\begin{definition}[Partial Derivative]
    The \textbf{partial derivative of \( f(x,y) \) with respect to x} is the usual x-derivative, but with \( y \) held
    constant, written as \( \frac{\partial f}{\partial x} \) or \( f_x \).
    
    Similarly, \( \frac{\partial f}{\partial y} \) or \( f_y \) is the usual y-derivative, but with \( x \) held constant.
\end{definition}

We can interpret \( \frac{\partial f}{\partial x}(p_0) \) as the rate of change (r.o.c.) of \( f \) at \( p_0 \) in the
+x-direction and \( \frac{\partial f}{\partial y}(p_0) \) as the r.o.c. of \( f \) at \( p_0 \) in the +y-direction.

The limit definition of the partial derivative is a more rigorous statement of the first defintion. However, this can help
extend the partial derivative to functions of more variables in an obvious way.
\[ f_x(a,b) = \lim_{h->0} \frac{f(a+h,b)-f(a,b)}{h} \]
\[ f_y(a,b) = \lim_{h->0} \frac{f(a,b+h)-f(a,b)}{h} \]



\end{document}