\documentclass[../main.tex]{subfiles}

\begin{document}

%%%%%%%%%%%%%%%%%%%%%%%%%%%%%%%%%%%%%%%%%%%%%%%%%%%%%%%%%%%%%%%%%%%%%%%%%%%%%%%%%%%%%%%%%%%%%%%%%%%%%%%%%%%%%%%

\section{Introduction to LODEs}

\begin{definition}[Linear Ordinary Differential Equation (LODE) of Order n]
    A \textbf{linear ordinary differential equation} of order n is an equation:
    \[ y^{(n)}+a_{n-1}(x)y^{(n-1)}+\dots+a_{1}(x)y'+a_{0}(x)y=b(x)  \]
    where $y^{(i)}$ is the ith derivative.
\end{definition}

\begin{example}[Hooke's Law]
    Hooke's Law states the force exerted by a spring is proportional to its displacement from equilibrium:
    \[ F = -ky \]
    This applies when displacement is small compared to the total range of the spring.
    (Fun Fact: Hooke stated this as an anagram riddle before officially publishing his discovery.)

    From Newton,
    \begin{gather*}
        F = m\frac{d^2y}{dt^2} \\
        \therefore \, \text{By substitution, } y^{(2)}+\omega^2y=0 \\
        \text{where } \omega =\sqrt{ \frac{k}{m} }
    \end{gather*}

    This describes the system. The solution is some \(y(t)\): what actually happens. Finding \(y(t)\) is our goal.

    \textit{Check:} \(y=c_{0}\cos \omega t+c_{1}\sin \omega t\) is a solution where \(c_{0},c_{1}\) are constants.
    \begin{align*}
        y^{(1)} & = -c_{0}\omega \sin \omega t+c_{1}\omega \cos \omega t \\
        y^{(2)} & = -c_{0}\omega^2\cos \omega t-c_{1}\omega^2\sin \omega t \\
        & = -\omega^2 y
    \end{align*}
    Simple harmonic motion!
\end{example}

\begin{definition}[General Solution of a LODE]
    A \textbf{general solution of a LODE} is a solution of the form:
    \[ y = c_{1}y_{1}+c_{2}y_{2}+\dots+c_{n}y_{n} \]
    where the \(y_{i}\) are independent solutions and \(c_{j}\) are unknowns.
\end{definition}

\begin{note}
    Any solution is some specialization of the \(c_{j}\)'s
\end{note}

\begin{example}[Newton's Law of Cooling]
    Newton's Law of Cooling is as follows: \[ \frac{dT}{dt}=-k(T-T_{m})\]
    where T = temperature at time t, t = time, k = constant, and \(T_{m}\) = ambient temp

    \textit{Check:} \( T=T_{m}+ce^{-kt} \) is a solution. Witness:
    \begin{align*}
        \frac{dT}{dt} &= -kce^{-kt} \\
        &= -k(T-T_{m})
    \end{align*}
\end{example}

%%%%%%%%%%%%%%%%%%%%%%%%%%%%%%%%%%%%%%%%%%%%%%%%%%%%%%%%%%%%%%%%%%%%%%%%%%%%%%%%%%%%%%%%%%%%%%%%%%%%%%%%%%%%%%%

\section{Other Examples of Differential Equations}

Here are additional examples of differential equations explored more throughly.

\subsection{Orthogonal Trajectory Problem}

Given a function \( f(x,y) \), there is a family of level curves  \(f(x,y)= f(p),\, p=(a,b)\). We write the gradient of $f(x,y)$ as $<f_{x},f_{y}>$.

\begin{note}
    The gradient has the following important properties:
    \begin{itemize}
        \item The direction is in the greatest rate of change of \( f \)
        \item \( \vec{\nabla}f \perp {f(x,y)=f(p)} \)
    \end{itemize}
\end{note} 

The problem is thus to find a curve that follows the gradient i.e. an \textbf{orthogonal trajectory curve}.

\vbox{
    \textbf{Potential Applications:}
    \begin{itemize}
        \item Heat seaking missiles, \( f =\) temperature
        \item Charged particle, \( f =\) electric potential
        \item Bear at scout jamboree, \( f =\) peanut butter scent intensity
    \end{itemize}
}


\textbf{Theory:}
Let \( g(x,y)=g(p) \) be an \textbf{orthogonal trajectory curve} (in xy-plane).

\textit{Goal:} Find g.

\textit{Require:} \( \vec{\nabla}f(p)\perp\vec{\nabla}g(p) \)

Then, it follows that \( f_{x}g_{x}+f_{y}g_{y}=0 \)

Therefore the slope of D at p is the slope of the vector \( \displaystyle \left< -g_{y},g_{x} \right> =-\frac{g_{x}}{g_{y}}=\frac{f_{y}}{f_{x}} \). \\
Therefore \( \displaystyle \frac{dy}{dx}=\frac{f_{y}}{f_{x}} \) is a 1st order LODE for g.

To find g, use the relation \( \displaystyle dy=\frac{f_{y}}{f_{x}}dx \) then integrate.

\begin{example}[]
    \[ f(x,y)=x^2+y^2 \]
    Level curves are circles
    \begin{gather*}
        \frac{f_{y}}{f_{x}}=\frac{y}{x} \\
        \therefore dy=\frac{y}{x}dx \\
        \int \frac{1}{y} \, dy = \int \frac{1}{x} \, dx \\
        \therefore \ln y=\ln x + C \\
        y=kx
    \end{gather*}
    These are all lines through (0,0).
    Compute these through \( \displaystyle p=(a,b),k=\frac{b}{a} \).
\end{example}

\subsection{Initial Value Problems}

Most general LODE of order n is:
\[ y^{(n)}+a_{1}(x)y^{(n-1)}+\dots+a_{n}(x)y=F(x) \]
where \( F(x) \) is the driving term.

\vbox{
    An \textbf{initial value problem (IVP)} is an equation, \( (*) \) together with initial conditions, \( (* *) \):
    \begin{gather*}
        y(x_{0})=y_{0} \\
        y^{(1)}(x_{0})=y_{1} \\
        y^{(n-1)}(x_{0})=y_{n-1}
    \end{gather*}
}


\begin{example}[]
    Solve the IVP \( y''+\omega^2y=0, \space y(0)=1,\space y'(0)=0 \)
    \begin{gather*}
        \text{The general solution is: } y=c_{0}\cos(\omega t)+c_{1}\sin(\omega t) \\
        \text{Initial conditions impose:} \\
        c_{0}\cos0+\overset{0}{\msout{c_{1}\sin0}}=1 \implies c_0 = 1 \\
        \overset{0}{\msout{-c_0\omega\sin0}}+c_1\omega\cos0=0 \implies c_1 = 0 \\
        \therefore \boxed{y=\cos \omega t}
    \end{gather*}
    In general, \( c_0 = y(0),\,c_1 = \frac{y'(0)}{\omega} \).
\end{example}

\begin{theorem}[]
    Assume in \( (*) \) that the \(a_{i}(x) \) and \( F(x) \) are continuous on some interval \( I
     \).
    Then there exists a unique solution to IVP \( (*) \) + \( (**) \) on \( I \).
\end{theorem}

%%%%%%%%%%%%%%%%%%%%%%%%%%%%%%%%%%%%%%%%%%%%%%%%%%%%%%%%%%%%%%%%%%%%%%%%%%%%%%%%%%%%%%%%%%%%%%%%%%%%%%%%%%%%%%%

\section{First Order DE's}

The form of a \textbf{first order differential equation} is: \( y' = f(x,y) \). Does it have a solution? If so, is it unique?

\begin{theorem}[]
    Given \( y'=f(x,y) \). Assume \( f(x,y) \) is continuous on \( [a,b] \times [c,d] \) and \( f_{y} \) is continuous in \( (a,b) \times (c,d) \).
    Then for each \( (x_0,y_0) \in (a,b) \times (c,d),\, \) there exists a unique solution to \( y=y(x) \) running through \( (x_0,y_0) \) and this solution holds for some interval \( I \) around \( x_0 \).
\end{theorem}

\begin{note}
    The notation \( [a,b] \times [c,d] \) describes the domain and range using the Cartesian product of the two sets. In other words, \(a \leq x \geq b \) and \( c \leq y \geq d \).
\end{note}

\begin{example}[]
    Is there a unique solution to: \( y'=xy^{\frac{1}{2}},\,y(0)=0 \)?

    \textit{Sol'n.}
    \( f(x,y)=xy^{\frac{1}{2}} \) is continuous on \( (-\infty,\infty)\times[0,\infty] \)
    \( f_y = \frac{1}{2}xy^{-\frac{1}{2}} \) is not continuous at y = 0.
    
    Therefore the theorem is useless.

    \vbox{
        There are solutions for the IVP where y(0) = 0. Check it out:
        \begin{itemize}[mode=unboxed]
            \item \( y=0 \) is a solution. \( y(0) = 0 \)
            \item \( y=\frac{1}{16}x^4 \) is also a solution.
                \begin{gather*}
                    y(0) = 0 \\
                    y'=\frac{1}{4}x^3=xy^{\frac{1}{2}} \checkmark
                \end{gather*}
        \end{itemize}
        Thus the solutions are not unique as expected from the theorem.
    }
\end{example}

\begin{note}
    Slope fields can be used to visualize solutions to a differential equation.
\end{note}

\begin{definition}[Equilibrium Solutions]
    \textbf{Equilibrium solutions} are solutions to a differential equation that have a derivative of zero everywhere i.e. equal to a constant value.
\end{definition}

In terms of a first order linear differential equation, the equilibrium solutions are the set \( \{ y = y_0 \mid f(x,y_0)=0 \} \)
On slope field diagrams, they are where a horizontal line fits as a solution.

%%%%%%%%%%%%%%%%%%%%%%%%%%%%%%%%%%%%%%%%%%%%%%%%%%%%%%%%%%%%%%%%%%%%%%%%%%%%%%%%%%%%%%%%%%%%%%%%%%%%%%%%%%%%%%%

\subsection{Analytical Techniques for First Order ODEs}

\begin{enumerate}[mode=unboxed]
    \item \textbf{Seperable:} If the first order ODE is one of the form \( p(y)y'=q(x) \) (or \( y'=r(y)q(x) \), ...)
        then since \( dy=y'dx \), we get \( p(y)dy=q(x)dx \) and can integrate (in principle).
    \item \textbf{Integrating Factor:} For LODE's in the form, \( y'+p(x)y=q(x) \), let \( I(x)= e^{\int p(x) \,dx} \) be the integrating factor.

        \textit{Claim.} \( y(x)= \frac{1}{I(x)}\int q(x) I(x) \,dx \) is the general solution.
        Let's test this claim using the chain rule.
        \begin{align*}
            I(x)y(x)&=\int q(x) I(x) \,dx \\
            \frac{d}{dx}I(x)y(x)&=I'(x)y(x)+I(x)y'(x) \\
            &= p(x)I(x)y(x)+I(x)y'(x) \\
            &= I(x)(p(x)y(x)+y'(x))) \\
            &= I(x)q(x) \text{by original LODE}
        \end{align*}
        \( \therefore I(x)y(x)=\int I(x)q(x) \,dx \implies\) claim is true 

        \begin{note}
            The indefinite integral and the lack of a integration constant is not formally correct.
            However, the constant is removed upon the simplification.
            We can always divide both sides of the equation by \( e^{C} \).
        \end{note}
\end{enumerate}

An \textbf{autonoumous ODE} is one with no independent variable. It is usually seperable.

\begin{example}[]
    Find the general solution for \( y'+ ( \tan x ) y=\cos^{2}x \)  on \( -\pi/2,\, \pi/2 \).

    \textit{Sol'n.} Recognize this is the form for the integrating factor with \( p(x)=\tan x \) and \( q(x) = \cos^{2}x \).
    \begin{align*}
        I(x)= e^{\int \tan x \,dx} = e^{\ln(\sec x)} = \sec x \\
        (\sec x) y = \int \cos (x) \,dx = \sin x + C \\
        y = \sin x \cdot \cos x + C \cdot \cos x
    \end{align*}
\end{example}

%%%%%%%%%%%%%%%%%%%%%%%%%%%%%%%%%%%%%%%%%%%%%%%%%%%%%%%%%%%%%%%%%%%%%%%%%%%%%%%%%%%%%%%%%%%%%%%%%%%%%%%%%%%%%%%

\subsection{Application: Revisiting Newton's Law of Cooling}

Newton defined his Law of Cooling as:
\[ T' = -k(T-T_m) \]
In its LODE form, \( dT = -k(T-T_m)dt \). (Now, \( T_m \) could be time-dependent.)

\begin{example}[]
    Suppose \( k=\frac{1}{40},\, T_m(t)=80e^{-\frac{t}{20}},\,T(0)=0 \).
    \begin{enumerate}[label=\alph*), nolistsep]
        \item Solve this IVP.
        \item Determine the asymptotic behavior of the solution.
        \item Find the maximum temperature.
    \end{enumerate}

    \begin{enumerate}[mode=unboxed,label=\alph*)]
        \item Write it out as an ODE: \[ T'=-\frac{1}{40}(T-80e^{-\frac{t}{20}}) \]
        Not seperable, 1st order ODE. This is an integrating factor problem.
        In standard form:
        \begin{gather*}
            T' + \underbrace{\frac{1}{40}}_{p(t)} T = \underbrace{2e^{-\frac{t}{20}}}_{q(t)} \\
            \implies p(t) = \frac{1}{40} \text{ and } q(t) = 2e^{-\frac{t}{20}} \\
            \\
            I(t) = e^{\int p(t) \,dt} \\
            I(t) = e^{\int \frac{1}{40} \,dx} = e^{t/40} \\
            \\
            \text{Multiplying the original LODE by } I(t) \text{:}\\
            \underbrace{I(t)T' + I(t)\frac{1}{40}T}_{(IT)'} = \underbrace{I(t)2e^{-\frac{t}{20}}}_{2e^{-\frac{t}{40}}} \\
            IT=\int 2e^{-\frac{t}{40}} \,dt = -80e^{-\frac{t}{40}}+C \\
            \therefore T = e^{-t/40} \left[-80e^{-\frac{t}{40}}+C \right] \\
            \\
            \text{Using IV: } T(0)=0 \implies 1[-80+C]=0 \\
            C=80 \\
            \therefore T = e^{-t/40} \left[ -80e^{-\frac{t}{40}}+80 \right] \\
            \boxed{T=80\left(-e^{-\frac{t}{20}}+e^{-\frac{t}{40}}\right)}
        \end{gather*}
        \item \( \boxed{ \lim_{t\to\infty}T(t)= 0 } \)
        \item Maximum temperature? Occurs when \( \frac{dT}{dt} =0\) or \( T(t)=T_{m}(t) \)
            \begin{align*}
                T(t) &= T_m(t) \\
                80(e^{-\frac{t}{40}}-e^{-\frac{t}{20}}) &= 80e^{-\frac{t}{20}} \\
                e^{-\frac{t}{40}}-e^{-\frac{t}{20}} &= e^{-\frac{t}{20}} \\
                e^{-\frac{t}{40}} &= 2e^{-\frac{t}{20}} \\
                -\frac{t}{40} &= \ln 2 - \frac{t}{20} \\
                \frac{t}{40} &= \ln 2 \\
                t &= 40\ln 2
            \end{align*}
            So \( T_{max} = 80\left( \frac{1}{2}-\frac{1}{4} \right) = \boxed{20} \)
    \end{enumerate}
\end{example}

%%%%%%%%%%%%%%%%%%%%%%%%%%%%%%%%%%%%%%%%%%%%%%%%%%%%%%%%%%%%%%%%%%%%%%%%%%%%%%%%%%%%%%%%%%%%%%%%%%%%%%%%%%%%%%%

\subsection{Application: RLC Circuits}

\textbf{The Physics:}
\begin{itemize}[mode=unboxed]
    \item Voltage is the difference \( \Delta V \) in electric potential \( V \)
    \item Voltage along a wire (usually caused by a battery or EMF) causes charge \( q \) to move,
        causing a current \( \displaystyle i = \frac{dq}{dt} \)
    \item \textbf{Kirchoff's Second Law:}
        \[ \sum_{\text{closed loop}} \Delta V = 0 \]
        In a circuit, we have components with opposing voltages.
        \begin{enumerate}
            \item Resistance \( R \) causes \( \overset{\text{(Ohm's Law)}}{\Delta V = iR} \). Units are ohms \( \Omega \).
            \item Capacitance \( C \) causes \( \displaystyle \Delta V = \frac{1}{C}q \). Units are farads \( F \).
            \item Inductance \( C \) causes \( \displaystyle \Delta V = L\frac{di}{dt} \). Units are henrys \( H \).
            \item Driver EMF \( E(t)=\sum_{\text{circuit components}} \Delta V \).
        \end{enumerate}
\end{itemize}

\begin{aside}
    The work done to move a unit charge from A to B can be found using multivariable calculus!
    \begin{align*}
        int_C \vec{E} \cdot \vec{T} \,ds & = \int_C \vec{\Delta V} \cdot \vec{T} \,ds \\
        & \overset{FTC}{=} V(b)-V(a)
    \end{align*}
\end{aside}

By Kirchoff's Law, we have an LODE:
\[ L\frac{di}{dt} + Ri + \frac{1}{C}q = \frac{1}{L} E(t) \]
Since \( i=\frac{dq}{dt} \),
\[ \boxed{ \frac{d^{2}q}{{dt}^2} + \frac{R}{L}\frac{dq}{dt} + \frac{1}{LC}q = \frac{1}{L}E(t) } \]

Generally, either \( E(t) = E_0 \) (constant as DC) or \( E(t)= E_0 \cos \omega t \) (AC, often 60 Hz).
To solve this equation, we must restrict it to either an RL case (first order LODE in i) or RC case (first order LODE in q).

\begin{example}[RL Circuit]
    Given \( E(t)=E_0 \cos \omega t \). Find \(i(t), \text{ given } i(0) = 0 \).

    \textit{Sol'n.} LODE is \( \displaystyle \frac{di}{dt} + \overset{p(t)}{\overbrace{\frac{R}{L}}}i = \overset{q(t)}{\overbrace{\frac{E_0}{L} \cos \omega t}} \).
    Not seperable, use integrating factor.

    \[ I(t) =e^{\int \frac{R}{L} \,dt}=e^{at} \, \left( a = \frac{R}{L} \right) \]
    \[ e^{at}i(t) = \int e^{at}E_0 \cos \omega t \,dt \]
    Requires integration by parts

    \[ \boxed{i(t) = \underset{\text{steady state}}{\underbrace{\frac{E_0a}{L(a^2+\omega^2)}\left( a \cos \omega t + \omega \sin \omega t\right)}} +
        \underset{\text{transient}}{\underbrace{\frac{E_0a}{L(a^2+\omega^2)}\left( -ae^{-at} \right)}}} \]
\end{example}

\end{document}