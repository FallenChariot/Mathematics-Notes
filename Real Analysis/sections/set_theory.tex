\documentclass[../main.tex]{subfiles}

\begin{document}

\section{Set Theory Foundation}

This is a review of some introductory proof concepts that are important for laying the foundations of both
rational numbers and real analysis.

\begin{definition}[]
    A \textbf{set} is a collection of objects called elements of the set.
\end{definition}

\textit{Example:}
\begin{enumerate}
    \item \( S = \{ 1, 2, 3 \} \; (= \{ 1,2,3,3 \}) \)
    \item \( E = \{ \text{Even integers } \} \)
    \item \( \{ \text{College students} \} \)
\end{enumerate}

\textit{Notation:}
\begin{itemize}
    \item \( x \in S \) means \( x \) is in \( S \).
    \item \( x \notin S \) means \( x \) is not in \( S \).
    \item The empty set \( \emptyset \) is the set with no elements.
    \item \( A \subseteq B \) means \( A \) is a subset of B (i.e. if \( x \in A \), then \( x \in B \)).
    \item If \( A \subseteq B \) but \( B \subsetneq A \) A is a proper subset.
\end{itemize}

If \( A \subseteq B \) and \( B \subseteq A \) then \( A = B \). Otherwise \( A \neq B \).

We can define more sets in terms of other sets.
\textit{Set Operations:}
Let \( A \text{ and } B \) be sets.
\begin{itemize}
    \item Union: \( A \cup B  = \{ x \, | \, x \in A \text{ or } x \in B \}\)
    \item Intersection: \( A \cap B  = \{ x \, | \, x \in A \text{ and } x \in B \}\)
    \item Compliment: \( B - A = \{ x \, | \, x \in B \text{ and } x \notin A \} \)
    \item Product: \( A \times B = \{ (a, b) \, | \, a \in A \text{ and } b \in B \} \)
\end{itemize}

If \( U \) is a universal set (set of everything in context), we write
\( \bar{A} = U - A \\ = \{ x \, | \, x \in U \text{ and } x \notin A \} \).

\end{document}