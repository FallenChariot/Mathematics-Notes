\documentclass[../main.tex]{subfiles}

\begin{document}

\section{Constructing the Real Numbers}

%%%%%%%%%%%%%%%%%%%%%%%%%%%%%%%%%%%%%%%%%%%%%%%%%%%%%%%%%%%%%%%%%%%%%%%%%%%%%%%%%%%%%%%%%%%%%%%%%%%%%%%%%%%%%%%

\subsection{Upper Bounds}

Now, we seen in the previous section that \( \mathbb{Q} \) has ``gaps".
\( x^2 = 2 \) has no solution in \( \mathbb{Q} \).
\vspace{5mm}
\begin{center}
    \begin{tikzpicture}[]
        \draw[thick,densely dotted, <->, xscale=2] (-3.5,0) -- (3.5,0) node[anchor=north west] {};
        \foreach \x in {-3, -2, -1, 0, 1, 2, 3}
            \draw[xscale=2] (\x cm,2pt) -- (\x cm,-2pt) node[anchor=north] {\( \x \)};
        \draw[fill=white] (2.8,0)  circle [radius=1.5pt]  node[below] {\( x \)};
    \end{tikzpicture}
\end{center}

We need to fill in these gaps somehow while not knowing where the gaps and holes are.

\begin{definition}[Upper Bound]
    Let \( E \subset S \) ordered.
    If there exists \( \beta \in S \) such that for all \( x \in E \), \( x \leq \beta \),
    then \( \beta \) is an \textbf{upper bound (u.b.)} for \( E \).
    We say \( E \) is bounded above.
\end{definition}

A lower bound can be defined similarly with ``greater than or equal to."

\begin{example}[]
    Consider the set \( A = \left\{ x \mid x^2 < 2 \right\} \). \( 2 \) is an u.b. for A.
    \( \displaystyle \frac{2}{3} \) is also an u.b. for \( A \).
\end{example}

\begin{definition}[Least Upper Bound]
    If there exists an \( \alpha \in S \) such that:
    \begin{enumerate}
        \item \( \alpha \) is an upper bound of \( E \)
        \item If \( \gamma < \alpha \), then \( \gamma \) is not an upper bound of E.
    \end{enumerate}
    Then \( \alpha \) is called a \textbf{least upper bound (lub)} of \( E \) or the \textbf{suprenum} of \( E \).
    Write \( \alpha = \text{sup} E \).
\end{definition}

\begin{example}[]
    Let \( S = \mathbb{Q} \).
    \begin{enumerate}
        \item \( E = \left\{ \displaystyle \frac{1}{2},\, 1,\, 2 \right\} \; \boxed{ \text{sup}E= 2 } \)
        \item \( E = \{x \in \mathbb{Q} \mid x < 0 \} \; \boxed{\text{sup}E = 0}\)
        \item \( E = \mathbb{Q} \; \boxed{\text{sup}E \text{ does not exist}} \)
        \item \( E = A \; (\text{as defined above}) \; \boxed{\text{sup}E \text{ does not exist}} \)
    \end{enumerate}
\end{example}

\begin{definition}[Least Upper Bound Property]
    A set \( S \) has the \textbf{least upper bound property} if every nonempty subset of \( S \) that has an upper bound has a least upper bound.
\end{definition}

%%%%%%%%%%%%%%%%%%%%%%%%%%%%%%%%%%%%%%%%%%%%%%%%%%%%%%%%%%%%%%%%%%%%%%%%%%%%%%%%%%%%%%%%%%%%%%%%%%%%%%%%%%%%%%%

\subsection{Dedekind Cuts}

\begin{definition}[Dedekind Cut]
    A \textbf{Dedekind cut} \( \alpha \) is a subset of \( \mathbb{Q} \) such that:
    \begin{enumerate}
        \item \( \alpha  \neq \emptyset, \, \mathbb{Q} \)
        \item If \( p \in \alpha \), \( q \in \mathbb{Q} \) and \( q < p \), then \( q \in \alpha \). (Closed downward)
        \item If \( p \in \alpha \), then \( p < r \) for some \( r \in \alpha \). (No largest number)
    \end{enumerate}
\end{definition}

\begin{example}[]
    \( \alpha = \left\{ x \in \mathbb{Q} \mid x < 0 \right\} \) is a cut.
\end{example}
\begin{enumerate}
    \item \( \alpha \neq 0, \mathbb{Q} \) \checkmark
    \item Let \( p \in \alpha \), \( q \in \mathbb{Q} \). Assume \( q < p \).
    By the transitivity property of order, \( q < 0 \).
    Thus, \( p \in \alpha \). \checkmark
    \item Let \( p \in \alpha \) and \( r \in \alpha \) such that \( r = \displaystyle \frac{q}{2} \).
    Since \( q < 0 \), \( q < \displaystyle \frac{q}{2} \). Thus \( q < r \). \checkmark
\end{enumerate}

\begin{example}[]
    \( \gamma = \left\{ r \mid r \leq 2 \right\} \) is not a cut.
    This set does have a largest element, 2.
\end{example}

\begin{definition}[Rational Numbers]
    Let \( \mathbb{R} = \left\{ \alpha \mid \alpha \text{ is a cut } \right\}\).
\end{definition}

We also define the following:
\begin{itemize}
    \item \( \alpha < \beta \) to mean \( \alpha \subsetneqq \beta \). This is an order.
    \item \( \alpha + \beta = \left\{ r + s \mid r \in \alpha \text{ and } s \in \beta \right\} \). (This means \( \mathbb{R} \) is a field.)
    \item \( \alpha \cdot \beta \)
    \begin{enumerate}
        \item For positive cuts \( \left\{ \alpha \mid \alpha > 0^* \right\} = \mathbb{R}_+ \): \\
            If \( \alpha, \beta \in \mathbb{R}_+ \), let \( \alpha \cdot \beta = \left\{ p \mid p < rs \, \text{for some } r \in \alpha, s \in \beta, \, and  r,s>0\right\} \).
        \item For cases with negative cuts, \( \alpha \cdot \beta =
        \begin{cases}
            (-\alpha)(-\beta) & \text{if } \alpha < 0^*, \, \beta < 0^* \\
            -[(-\alpha)\beta] & \text{if } \alpha < 0^*, \, \beta > 0^* \\
            -[\alpha(-\beta)] & \text{if } \alpha > 0^*, \, \beta < 0^*
        \end{cases} \) \\
        where the products are the same as defined for postive cuts and \( -\alpha, \, -\beta \) are the additive inverses of \( \alpha, \, \beta \) respectively.
        (The additive inverses are defined below.)
    \end{enumerate}
\end{itemize}

\begin{theorem}[]
    \label{thm_R}
    \( \mathbb{R} \) is an ordered field with the least upper bound property.
    \( \mathbb{R} \) contains \( \mathbb{Q} \) as a subfield.
\end{theorem}

%%%%%%%%%%%%%%%%%%%%%%%%%%%%%%%%%%%%%%%%%%%%%%%%%%%%%%%%%%%%%%%%%%%%%%%%%%%%%%%%%%%%%%%%%%%%%%%%%%%%%%%%%%%%%%%

\subsubsection{Proofs Showing \( \mathbb{R} \) is an Ordered Field}

The following section proves Theorem \ref{thm_R}.

\textit{Proof.}

\emph{Step 1:} We must show there is order on \( \mathbb{R} \). \\
Let \( \alpha,\beta,\gamma \in \mathbb{R} \).
We must show that they demonstrate both trichotomy and transitivity.
\begin{enumerate}
    \item Trichotomy: \\
    It is clear that at most one of the following can be true: \( \alpha < \beta, \, \alpha = \beta , \, \beta < \alpha \).
    For example, if \( \alpha < \beta \), then \( \alpha \subsetneqq \beta \)
    and by the definition of a proper subset, \( \alpha \neq \beta \) and \( \beta \) is not a proper subset of \( \alpha \).
    
    To show at least one of them must be true, suppose the first two statements are false.
    Then \( \alpha \) is not a subset of \( \beta \).
    By definition of a proper subset, there exists some \( a \in \alpha \) such that \( a \notin \beta \).
    Consider some \( b \in \beta \). Since \( \beta \) is closed downward, \( b < a \).
    This also means \( b \in \alpha \) since \( \alpha \) is also closed downward.
    This shows that \( \beta \subsetneqq \alpha \). Thus \( \beta < \alpha \).
    We conclude that at least one of these statements must be true.

    \item Transitivity: \\
    We assume that \( \alpha < \beta \) and \( \beta < \gamma \).
    By definition of \( < \), \( \alpha \subsetneqq \beta \) and \( \beta \subsetneqq \gamma \).
    By definition of a proper subset, \( \alpha \subsetneq \gamma \) and we conclude that \( \alpha < \gamma \).
\end{enumerate}
We have shown the cuts demonstrate order.

\emph{Step 2:} Next, we must show that addition is closed (\textbf{A1}). \\
Let \( \alpha, \beta \in \mathbb{R} \) and \( \gamma = \{r+s \, | \, r \in \alpha \text{ and } b \in \beta \} \).
To show addition is closed, we must show that \( \gamma \) is a cut.
\begin{enumerate}
    \item First, we must show that \( \gamma \neq \emptyset, \, \mathbb{Q} \). \\
    It should be clear that \( \gamma \) cannot be the empty set.
    Since \( \alpha, \beta \neq \mathbb{Q} \), there exists \( a' \notin \alpha, \, b' \notin \beta \).
    Now \( a < a' \) and \( b < b' \) for all \( a \in \alpha, \, b \in \beta \). Thus \( a + b < a' + b' \).
    Therefore \( a' + b' \notin \gamma \). We conclude that \( \gamma \neq \emptyset, \mathbb{Q} \).

    \item Second, we must show that \( \gamma \) is closed downward. \\
    Let \( p \in \gamma, q \in \mathbb{Q} \). Assume \( q < p \).
    Since \( p \in  \gamma \), there exists \( r \in \alpha \) and \( s \in \beta \) such that \( p = r+s \) so \( q < r+s \).
    This means \( q-s < r \). Since \( \alpha \) is closed downward, \( q-s \in \alpha \).
    Then \( q = q-s +s \) where \( q-s \in \alpha \) and \( s \in \beta \). We have shown that \( \gamma \) is closed downward.

    \item Third, we must show that \( \gamma \) has no largest number. \\
    Let \( t \in \gamma \). Then there exists \( u \in \alpha \) and \( v \in \beta \) such that \( t = u + v \).
    Since both cuts \( \alpha, \beta \) have no largest number, there exists \( x \in \alpha \) where \( u < x \) and \( y \in \beta \) where \( v < y \).
    Thus \( x + y \in \gamma \) and \( t < x + y \). We have shown that \( \gamma \) has no largest member.
\end{enumerate}
We have shown that \( \gamma \) meets the definition of a cut.

\emph{Step 3:} \textbf{A2} and \textbf{A3} follows since addition in \( \mathbb{Q} \) is commutative and associative.
    
\emph{Step 4:} We must show that \( 0^* =\left\{ q \in \mathbb{Q} \mid q < 0 \right\}\) is the additive identity for \( \mathbb{R} \) (\textbf{A4}). \\
In other words, we must show that \( \alpha + 0^* = \alpha \).
Let \( \alpha \in \mathbb{R} \) and \( 0^* = \left\{ q \in \mathbb{Q} \mid q < 0 \right\} \).
\begin{enumerate}
    \item First, we must show that \( \alpha + 0^* \subset \alpha \). \\
    Let \( a \in \alpha \) and \( b \in 0^* \).
    Since \( b < 0 \), \( a + b < a \).
    Thus \( a + b \in \alpha \) (since \( \alpha \) is closed downward).
    We conclude that \( \alpha + 0^* \subset \alpha \).

    \item Second, we must show that \( \alpha \subset \alpha + 0^* \). \\
    Let \( x,y \in \alpha \) and \( z \in 0^* \). Since \( \alpha \) has no largest member, we can pick \( x,y \) such that \( y > x \).
    Then \( x - y \in 0^* \) and similar to before \( x = y + (x - y) \in \alpha + 0^* \).
    We conclude that \( \alpha \subset \alpha + 0^* \).
\end{enumerate}
Thus we conclude that \( \alpha + 0^* = \alpha \).

\emph{Step 5:} Next, we must show that the additive inverse for \( \alpha \in \mathbb{R} \) is
\( \beta = \{ p \in \mathbb{Q} \, | \text{ there exists } \\ r > 0 \text{ s.t. } -p-r \notin \alpha \} \) (\textbf{A5}). \\
Let \( \alpha \in \mathbb{R} \) and \( \beta = \{ p \in \mathbb{Q} \, | \text{ there exists } r > 0 \text{ s.t. } -p-r \notin \alpha \} \).
\begin{enumerate}
    \item First, we must show that \( \beta \) is a cut.
    \begin{enumerate}[label=\roman*.]
        \item We must show that \( \beta \) is non-trivial. \\
        Since \( \alpha \neq \emptyset \), there exists some \( a \in \alpha \).
        Since \( \alpha \) is closed downward, \( a-b \in \alpha \) for all \( b > 0 \).
        Thus \( -a \notin \beta \) so \( \beta \neq \mathbb{Q} \).

        Since \( \alpha \in \mathbb{Q}\), there exists some \( c \notin \alpha \).
        Consider \( d = -c-1 \). Then \( -d-1 \notin \alpha \) (since \( c = -d -1 \)).
        Thus \( d \in \beta \) so \( \beta \neq \emptyset \).

        \item We must show that \( \beta \) is closed downward. \\
        Let \( f \in \beta, \, g \in \mathbb{Q} \). We assume that \( g < f \).
        There exists some \( h > 0 \) s.t. \( -f-h \notin \alpha \).
        Since \( g < f \), \( -g-h > -f-h \). Thus since \( \alpha \) is closed downward, \( -g-h \notin \alpha \).
        We conclude that \( \beta \) is closed downward.

        \item We must show that \( \beta \) has no largest member. \\
        Consider \( j = f + (h/2) \). Then \( j > f \), and \( -j - (h/2) = -f - h \notin \alpha \). Then \( j \in \beta \).
        Thus we find \( \beta \) also has no largest member.
    \end{enumerate}
    We conclude that \( \beta \) satisfies the definition of a cut.

    \item Second, we must show that \( \alpha + \beta = 0^* \).
    \begin{enumerate}[label=\roman*.]
        \item We need to show that \( \alpha + \beta \subset 0^* \). \\
        Let \( s \in \alpha \) and \( t \in \beta \). By definition of \( \beta \), there exists \( u > 0 \) s.t. \( -t-u \notin \alpha \).
        Then \( -t-u > s \) and it follows that \( s + t < -u < 0 \). Therefore \( s+t \in 0^* \).
        \item We also need to show that \( 0^* \subset \alpha + \beta \). \\
        Let \( v \in 0^* \) and \( w = -v/2 \). Then \( w > 0 \).
        By the Archimedean property of \( \mathbb{Q} \), there exists \( n \in \mathbb{Z} \) s.t. \( nw \in \alpha \) and \( (n+1)w \notin \alpha \).
        Consider \( x = -(n+2)w \). Now \( -x = nw+2w \implies -x-w = nw+w = (n+1)w \notin \alpha \). By definition of \( \beta \), \( x \in \beta \).
        We find \( v = -2w = -(n+2)w + nw = p + x \) and conclude that \( v \in \alpha + \beta \).
    \end{enumerate}
    This shows that \( \alpha + \beta = 0^* \).
\end{enumerate}
We have proved that \( \beta \) is the additive inverse for \( \alpha \) and that \( \mathbb{R} \) satisfies all the addition axioms.
Next we will show that \( \mathbb{R} \) satisfies the multiplication axioms.

\emph{Step 6:} We must show that \( \mathbb{R} \) is closed under multiplication.

\end{document}