\documentclass[../main.tex]{subfiles}

\begin{document}

\section{Constructing the Real Numbers}

%%%%%%%%%%%%%%%%%%%%%%%%%%%%%%%%%%%%%%%%%%%%%%%%%%%%%%%%%%%%%%%%%%%%%%%%%%%%%%%%%%%%%%%%%%%%%%%%%%%%%%%%%%%%%%%

\subsection{Upper Bounds}

Now, we seen in the previous section that \( \mathbb{Q} \) has ``gaps".
\( x^2 = 2 \) has no solution in \( \mathbb{Q} \).
\vspace{5mm}
\begin{center}
    \begin{tikzpicture}[]
        \draw[thick,densely dotted, <->, xscale=2] (-3.5,0) -- (3.5,0) node[anchor=north west] {};
        \foreach \x in {-3, -2, -1, 0, 1, 2, 3}
            \draw[xscale=2] (\x cm,2pt) -- (\x cm,-2pt) node[anchor=north] {\( \x \)};
        \draw[fill=white] (2.8,0)  circle [radius=1.5pt]  node[below] {\( x \)};
    \end{tikzpicture}
\end{center}

We need to fill in these gaps somehow while not knowing where the gaps and holes are.

\begin{definition}[Upper Bound]
    Let \( E \subset S \) ordered.
    If there exists \( \beta \in S \) such that for all \( x \in E \), \( x \leq \beta \),
    then \( \beta \) is an \textbf{upper bound (u.b.)} for \( E \).
    We say \( E \) is bounded above.
\end{definition}

A lower bound can be defined similarly with ``greater than or equal to."

\begin{example}[]
    Consider the set \( A = \left\{ x \mid x^2 < 2 \right\} \). \( 2 \) is an u.b. for A.
    \( \displaystyle \frac{2}{3} \) is also an u.b. for \( A \).
\end{example}

\begin{definition}[Least Upper Bound]
    If there exists an \( \alpha \in S \) such that:
    \begin{enumerate}
        \item \( \alpha \) is an upper bound of \( E \)
        \item If \( \gamma < \alpha \), then \( \gamma \) is not an upper bound of E.
    \end{enumerate}
    Then \( \alpha \) is called a \textbf{least upper bound (lub)} of \( E \) or the \textbf{suprenum} of \( E \).
    Write \( \alpha = \text{sup} E \).
\end{definition}

\begin{example}[]
    Let \( S = \mathbb{Q} \).
    \begin{enumerate}
        \item \( E = \left\{ \displaystyle \frac{1}{2},\, 1,\, 2 \right\} \; \boxed{ \text{sup}E= 2 } \)
        \item \( E = \{x \in \mathbb{Q} \mid x < 0 \} \; \boxed{\text{sup}E = 0}\)
        \item \( E = \mathbb{Q} \; \boxed{\text{sup}E \text{ does not exist}} \)
        \item \( E = A \; (\text{as defined above}) \; \boxed{\text{sup}E \text{ does not exist}} \)
    \end{enumerate}
\end{example}

\begin{definition}[Least Upper Bound Property]
    A set \( S \) has the \textbf{least upper bound property} if every nonempty subset of \( S \) that has an upper bound has a least upper bound.
\end{definition}

%%%%%%%%%%%%%%%%%%%%%%%%%%%%%%%%%%%%%%%%%%%%%%%%%%%%%%%%%%%%%%%%%%%%%%%%%%%%%%%%%%%%%%%%%%%%%%%%%%%%%%%%%%%%%%%

\subsection{Dedekind Cuts}

\begin{definition}[Dedekind Cut]
    A \textbf{Dedekind cut} \( \alpha \) is a subset of \( \mathbb{Q} \) such that:
    \begin{enumerate}
        \item \( \alpha  \neq \emptyset, \, \mathbb{Q} \)
        \item If \( p \in \alpha \), \( q \in \mathbb{Q} \) and \( q < p \), then \( q \in \alpha \). (Closed downward)
        \item If \( p \in \alpha \), then \( p < r \) for some \( r \in \alpha \). (No largest number)
    \end{enumerate}
\end{definition}

\begin{example}[]
    \( \alpha = \left\{ x \in \mathbb{Q} \mid x < 0 \right\} \) is a cut.
    \begin{proof}[Proof.]
        \textit{Step 1:} \( \alpha \neq 0, \mathbb{Q} \) \checkmark \\
        \textit{Step 2:} Let \( p \in \alpha \), \( q \in \mathbb{Q} \). Assume \( q < p \).
        By the transitivity property of order, \( q < 0 \).
        Thus, \( p \in \alpha \). \checkmark \\
        \textit{Step 3:} Let \( p \in \alpha \) and \( r \in \alpha \) such that \( r = \displaystyle \frac{q}{2} \).
        Since \( q < 0 \), \( q < \displaystyle \frac{q}{2} \). Thus \( q < r \). \checkmark
    \end{proof}
\end{example}

\begin{example}[]
    \( \gamma = \left\{ r \mid r \leq 2 \right\} \) is not a cut.
    This set does have a largest element, 2.
\end{example}

\begin{definition}[Rational Numbers]
    Let \( \mathbb{R} = \left\{ \alpha \mid \alpha \text{ is a cut } \right\}\).
\end{definition}

We also define the following:
\begin{itemize}
    \item \( \alpha < \beta \) to mean \( \alpha \subsetneqq \beta \). This is an order.
    \item \( \alpha + \beta = \left\{ r + s \mid r \in \alpha \text{ and } s \in \beta \right\} \). (This means \( \mathbb{R} \) is a field.)
    \item \( \alpha \cdot \beta \)
    \begin{enumerate}
        \item For positive cuts \( \left\{ \alpha \mid \alpha > 0^* \right\} = \mathbb{R}_+ \): \\
            If \( \alpha, \beta \in \mathbb{R}_+ \), let \( \alpha \cdot \beta = \left\{ p \mid p < rs \, \text{for some} r \in \alpha, s \in \beta, \, r,s>0\right\} \).
        \item For cases with negative cuts, \( \alpha \cdot \beta =
        \begin{cases}
            (-\alpha)(-\beta) & \text{if } \alpha < 0^*, \, \beta < 0^* \\
            -[(-\alpha)\beta] & \text{if } \alpha < 0^*, \, \beta > 0^* \\
            -[\alpha(-\beta)] & \text{if } \alpha > 0^*, \, \beta < 0^*
        \end{cases} \) \\
        where the products are the same as defined for postive cuts.
    \end{enumerate}
\end{itemize}

\begin{theorem}[]
    \( \mathbb{R} \) is an ordered field with the least upper bound property.
    \( \mathbb{R} \) contains \( \mathbb{Q} \) as a subfield.
\end{theorem}

%%%%%%%%%%%%%%%%%%%%%%%%%%%%%%%%%%%%%%%%%%%%%%%%%%%%%%%%%%%%%%%%%%%%%%%%%%%%%%%%%%%%%%%%%%%%%%%%%%%%%%%%%%%%%%%

\subsubsection{Proofs of the Properties of \( \mathbb{R} \)}

\begin{proof}[Proof]
    (Show there is order on \( \mathbb{R} \).)
    Let \( \alpha,\beta,\gamma \) be cuts.

    \emph{Step 1:} (Trichotomy) \\
    It is clear that at most one of the following can be true: \( \alpha < \beta, \, \alpha = \beta , \, \beta < \alpha \).
    For example, if \( \alpha < \beta \), then \( \alpha \subsetneqq \beta \)
    and by the definition of a proper subset, \( \alpha \neq \beta \) and \( \beta \) is not a proper subset of \( \alpha \). \\
    Suppose the first two statements are false. Then \( \alpha \) is not a subset of \( \beta \).
    By definition of a (proper) subset, there exists \( a \in \alpha \) such that \( a \notin \beta \).
    If \( q \in \beta \), then \( q < p \) since \( p \notin \beta \).
    Since cuts are closed downward, \( q \in \alpha \) so \( \beta \subsetneqq \alpha \). Thus \( \beta < \alpha \).

    \emph{Step 2:} (Transitivity) \\
    Assume \( \alpha < \beta \) and \( \beta < \gamma \).
    By definition of \( < \), \( \alpha \subsetneqq \beta \) and \( \beta \subsetneqq \gamma \).
    By definition of a proper subset, \( \alpha \subsetneq \gamma \) and we conclude \( \alpha < \gamma \). \checkmark

    We have shown the cuts demonstrate order.
\end{proof}

\begin{proof}[Proof]
    (\textbf{A1:} Show addition is closed)
    Let \( \alpha, \beta \) be cuts and \( \gamma = \alpha + \beta \).

    \emph{Step 1:} (Show that \( \gamma \neq \emptyset, \mathbb{Q} \).) \\
    It should be clear that \( \gamma \) cannot be the empty set.
    Since \( \alpha, \beta \neq \mathbb{Q} \), there exists \( a' \notin \alpha, b' \notin \beta \).
    Consider \( a \in \alpha \) and \( b \in \beta \). Now \( a < a' \) and \( b < b' \). Thus \( a + b < a' + b' \).
    Therefore \( a' + b' \notin \gamma \). We conclude \( \gamma \neq \emptyset, \mathbb{Q} \). \checkmark

    \emph{Step 2:} (Show \( \gamma \) is closed downward.) \\
    Let \( p \in \gamma, q \in \mathbb{Q} \). Assume \( q < p \).
    Since \( p \in  \gamma \), there exists \( r \in \alpha \) and \( s \in \beta \) such that \( p = r+s \) so \( q < r+s \).
    This means \( q-s < r \). Since \( \alpha \) is closed downward, \( q-s \in \alpha \).
    Then \( q = q-s +s \) where \( q-s \in \alpha \) and \( s \in \beta \) as desired.

    \emph{Step 3:} (Show \( \gamma \) has no largest number.) \\
    Let \( t \in \gamma \). Then there exists \( u \in \alpha \) and \( v \in \beta \) such that \( t = u + v \).
    Since both cuts \( \alpha, \beta \) have no largest number, there exists \( x \in \alpha \) where \( u < x \) and \( y \in \beta \) where \( v < y \).
    Thus \( x + y \in \gamma \) and \( t < x + y \). \checkmark

    We have shown \( \gamma \) meets the definition of a cut.
\end{proof}

\textbf{A2} and \textbf{A3} follows since addition in \( \mathbb{Q} \) is commutative and associative.
    
\begin{proof}[Proof]
    (\textbf{A4:} Show \( 0^* =\left\{ q \in \mathbb{Q} \mid q < 0 \right\}\) is the additive identity for \( \mathbb{R} \). 
    In other words, show \( \alpha + 0^* = \alpha \).) 

    \emph{Step 1:} (Show \( \alpha + 0^* \subset \alpha \).) \\
    Let \( a \in \alpha \) and \( b \in 0^* \).
    Since \( b < 0 \), \( a + b < a \).
    Thus \( a + b \in \alpha \) (since \( \alpha \) is closed downward).
    We conclude \( \alpha + 0^* \subset \alpha \).

    \emph{Step 2:} (Show \( \alpha \subset \alpha + 0^* \).) \\
    Let \( x,y \in \alpha \) and \( z \in 0^* \). We can pick \( x,y \) such that \( y > x \). Then \( x - y \in 0^* \)
    and similar to before \( x = y + (x - y) \in \alpha + 0^* \).
    We conclude \( \alpha \subset \alpha + 0^* \). \checkmark

    We conclude \( \alpha + 0^* = \alpha \).
\end{proof}

\begin{proof}[Proof]
    (\textbf{A5:} Show )
\end{proof}

\end{document}