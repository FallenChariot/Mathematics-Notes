\documentclass[../main.tex]{subfiles}

\begin{document}

\subsection{Constructing the Real Numbers}

\subsubsection{Dedekind Cuts}

Now, we seen in the previous section that \( \mathbb{Q} \) has ``gaps".
\( x^2 = 2 \) has no solution in \( \mathbb{Q} \).
\vspace{5mm}
\begin{center}
    \begin{tikzpicture}[]
        \draw[thick,densely dotted, <->, xscale=2] (-3.5,0) -- (3.5,0) node[anchor=north west] {};
        \foreach \x in {-3, -2, -1, 0, 1, 2, 3}
            \draw[xscale=2] (\x cm,2pt) -- (\x cm,-2pt) node[anchor=north] {\( \x \)};
        \draw[fill=white] (2.8,0)  circle [radius=1.5pt]  node[below] {\( x \)};
    \end{tikzpicture}
\end{center}

We need to fill in these gaps somehow while not knowing where the gaps and holes are.

\begin{definition}[Upper Bound]
    Let \( E \subset S \) ordered.
    If there exists \( \beta \in S \) such that for all \( x \in E \), \( x \leq \beta \),
    then \( \beta \) is an \textbf{upper bound (u.b.)} for \( E \).
    We say \( E \) is bounded above.
\end{definition}

A lower bound can be defined similarly with ``greater than or equal to."

\begin{example}[]
    Consider the set \( A = \left\{ x \mid x^2 < 2 \right\} \). \( 2 \) is an u.b. for A.
    \( \displaystyle \frac{2}{3} \) is also an u.b. for \( A \).
\end{example}

\begin{definition}[]
    If there exists an 
\end{definition}

\end{document}