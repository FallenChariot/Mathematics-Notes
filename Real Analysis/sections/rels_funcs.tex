\documentclass[../main.tex]{subfiles}

\begin{document}

\section{Functions and Relations}

It is also important to define some types of relations and functions.

\begin{definition}[Relation]
    A (binary) \textbf{relation} \( R \) on a set S is a subset of \( S \times S \). If \( (a,b) \in R \), we write \( aRb \).
\end{definition}

\begin{example}[of relations]

    \begin{enumerate}
        \item \( L \) "loves" is a relation on \( P \times P \) (where \( P \) is a set of all people).
        \item The set \( R = \left\{ (0,0), \, (0,1), \, (2,2), \, (7,18) \right\} \) is a relation on \( \mathbb{Z}^+ \). We would write \( 0R0 \), \( 0R1 \), \( 2R2 \), and \( 7R18 \).
    \end{enumerate}
\end{example}

\begin{definition}[Equivalence Relation]
    An equivalence relation on a set \( S \) is a relation s.t.:
    \begin{enumerate}
        \item Reflexive: For each \( a \in S \), \( a \sim a \).
        \item Symmetric: For \( a,b \in S \), if \( a \sim b \), then \( b \sim a \).
        \item Transitive: For \( a,b,c \in S \), if \( a \sim b \) and \( b \sim a \)
    \end{enumerate}
\end{definition}

Functions in the general sense are also a type of relation.
\begin{definition}[Function]
    A \textbf{function}, F from a set \( A \) to a set \( B \) is a relation s.t.:
    if \( aFb \) and \( aFb' \) then \( b = b' \). \\
    This is a rule that assigns a unique \( a \in A \) to a unique \( b \in B \).
    Write \( F(a) = b \).
\end{definition}



\end{document}