\documentclass[../main.tex]{subfiles}

\begin{document}

\section{Set Theory}

\begin{definition}[]
    A \textbf{set} is a collection of objects called elements of the set.
\end{definition}

\textit{Example:}
\begin{enumerate}
    \item \( S = \{ 1, 2, 3 \} \; (= \{ 1,2,3,3 \}) \)
    \item \( E = \{ \text{Even integers } \} \)
    \item \( \{ \text{College students} \} \)
\end{enumerate}

\textit{Notation:}
\begin{itemize}
    \item \( x \in S \) means \( x \) is in \( S \).
    \item \( x \notin S \) means \( x \) is not in \( S \).
    \item The empty set \( \emptyset \) is the set with no elements.
    \item \( A \subseteq B \) means \( A \) is a subset of B (i.e. if \( x \in A \), then \( x \in B \)).
    \item If \( A \subseteq B \) but \( B \subsetneq A \) A is a proper subset.
\end{itemize}

If \( A \subseteq B \) and \( B \subseteq A \) then \( A = B \). Otherwise \( A \neq B \).

We can define more sets in terms of other sets.
\textit{Set Operations:}
Let \( A \text{ and } B \) be sets.
\begin{itemize}
    \item Union: \( A \cup B  = \{ x \, | \, x \in A \text{ or } x \in B \}\)
    \item Intersection: \( A \cap B  = \{ x \, | \, x \in A \text{ and } x \in B \}\)
    \item Compliment: \( B - A = \{ x \, | \, x \in B \text{ and } x \notin A \} \)
    \item Product: \( A \times B = \{ (a, b) \, | \, a \in A \text{ and } b \in B \} \)
\end{itemize}

If \( U \) is a universal set (set of everything in context), we write
\( \bar{A} = U - A \\ = \{ x \, | \, x \in U \text{ and } x \notin A \} \).

\section{Functions and Relations}

It is also important to define some types of relations and functions.

\subsection{Relations}

\begin{definition}[Relation]
    A (binary) \textbf{relation} \( R \) on a set S is a subset of \( S \times S \). If \( (a,b) \in R \), we write \( aRb \).
\end{definition}

\begin{example}[of relations]

    \begin{enumerate}
        \item \( L \) "loves" is a relation on \( P \times P \) (where \( P \) is a set of all people).
        \item The set \( R = \left\{ (0,0), \, (0,1), \, (2,2), \, (7,18) \right\} \) is a relation on \( \mathbb{Z}^+ \). We would write \( 0R0 \), \( 0R1 \), \( 2R2 \), and \( 7R18 \).
    \end{enumerate}
\end{example}

\begin{definition}[Equivalence Relation]
    An equivalence relation on a set \( S \) is a relation s.t.:
    \begin{enumerate}
        \item Reflexive: For each \( a \in S \), \( a \sim a \).
        \item Symmetric: For \( a,b \in S \), if \( a \sim b \), then \( b \sim a \).
        \item Transitive: For \( a,b,c \in S \), if \( a \sim b \) and \( b \sim a \)
    \end{enumerate}
\end{definition}

%%%%%%%%%%%%%%%%%%%%%%%%%%%%%%%%%%%%%%%%%%%%%%%%%%%%%%%%%%%%%%%%%%%%%%%%%%%%%%%%%%%%%%%%%%%%%%%%%%%%%%%%%%%%%%%

\subsection{Functions}

Functions in the general sense are also a type of relation.
\begin{definition}[Function]
    A \textbf{function}, \( F \) from a set \( A \) to a set \( B \) is a relation s.t.:
    if \( aFb \) and \( aFb' \) then \( b = b' \). \\
    This is a rule that assigns a unique \( a \in A \) to a unique \( b \in B \).
    Write \( f: A \rightarrow B \) and \( f(a) = b \).
\end{definition}

\end{document}