\documentclass[../main.tex]{subfiles}

\begin{document}

\section{The Rational Numbers}

Assume \( Z \), the integers, have arithmetic order.
What is \( \mathbb{Q} \)? Perhaps it's the set: \\
\( \left\{ \displaystyle \frac{m}{n} \, \middle| \, m,n \in \mathbb{Z}, n \neq 0 \right\} \).

However, what does that fraction notation actually mean?
When we first begin teaching fractions to children we talk about splitting things like cake into smaller pieces.
If we have a whole cake made of 3 slices, we can give one person a slice so they have \( \displaystyle \frac{1}{3} \) of the cake.
If we have a cake of 6 slices, we could give them 2 slices instead. They would have \( \displaystyle \frac{2}{6} \).
These two fractions are equivalent though! We need more rigor (this is mathmematics of course).

We say that the fractions as equivalent ordered pairs \( (1, 3) \sim (2,6) \).
These belong to the same \textbf{equivalence class}, \( \left[\displaystyle \frac{1}{3}\right] \).

\begin{definition}[Rational Numbers]
    The \textbf{rational numbers}, \( \mathbb{Q} \), is the set \( \left\{ \displaystyle \frac{m}{n} \, \middle| \, m,n \in \mathbb{Z}, n \neq 0 \right\} \)
    where \( \displaystyle \frac{m}{n} \) is an equivalence class of \( (m,n) \) with the relation \( (m,n) \sim (p,q) \) if \( mq = np \) and \( q,n \neq 0 \)
\end{definition}


\begin{proof}[Proof. Is \( \sim \) an equivalence relation?]
    Need to show \( \sim \) reflextive, symmetric, and transitive.

    \emph{Step 1} Reflective: Let \( (p,q) \in \mathbb{Q} \). Show \( (p,q) \sim (p,q) \) \\
    Since \( ab = ba \), \( (p,q) \sim (p,q) \) \checkmark
    
    \emph{Step 2} Symmetry: Let \( (p,q), (m,n) \in \mathbb{Q} \). Assume \( (p,q) \sim (m,n) \). Show \( (m,n) \sim (p,q) \).
    \setlength{\abovedisplayskip}{0pt}    
    \begin{flalign*}
        (p,q) \sim (m,n) &\implies pn = qm \\
        &\implies qm = pn \\
        &\implies mq = np \\
        &\implies (m,n) \sim (p,q) \, \checkmark &
    \end{flalign*}
    
    \emph{Step 3} Transitive: Let \( (p,q), (m,n), (a,b) \in \mathbb{Q} \). Assume \( (p,q) \sim (m,n) \) and \( (m,n) \sim (a,b) \). Show \( (p,q) \sim (a,b) \). \\
    Need cancellation law on \( \mathbb{Z} \): if \( ab = ac \) and \( a \neq 0 \) then \( b=c \). \\
    \( (p,q) \sim (m,n) \implies pn = qm \) and \( (m,n) \sim (a,b) \implies mb = na \)

    \textit{Case 1:} \( p = 0 \)
    \setlength{\abovedisplayskip}{0pt}  
    \begin{flalign*}
        p = 0 &\implies pn = qm = 0 \\
        &\implies m = 0 \text{ since } q \neq 0 \\
        &\implies mb = na = 0 \\
        &\implies a = 0 \text{ since } n \neq 0 \\
        &\implies pb = qa = 0 \\
        &\implies (p,q) \sim (a,b) \, \checkmark & 
    \end{flalign*}

    \textit{Case 2:} \( m = 0 \) \\
    Similar to Case 1. \checkmark

    \textit{Case 3:} \( p,m \neq 0 \) \\
    Multiplying \( pn = qm \) by \( ab \): \( ab(pn) = ab(qm) \). \\
    \(\implies na(pb) = mb(qa) \) \\
    \( \implies pb = qa \) by cancellation law (\( m \neq 0 \) and \( mb = na \))
    \( \implies (p,q) \sim (a,b) \) \checkmark   
\end{proof}

%%%%%%%%%%%%%%%%%%%%%%%%%%%%%%%%%%%%%%%%%%%%%%%%%%%%%%%%%%%%%%%%%%%%%%%%%%%%%%%%%%%%%%%%%%%%%%%%%%%%%%%%%%%%%%%

\subsection{Arithmetic (of Rationals)}

Our definitions of arithmetic on \( \mathbb{Q} \) be well-defined.
For example, we could define addition as follows:
\[ \frac{a}{b} + \frac{c}{d} = \frac{a+c}{b+d} \]
However,
\begin{gather*}
    \frac{1}{2} + \frac{1}{3} = \frac{2}{5} \\
    \frac{2}{4} + \frac{3}{7} = \frac{3}{7}
\end{gather*}
\( \displaystyle \frac{1}{2} \) and \( \displaystyle \frac{2}{4} \) are in the same equivalent class, but \( \frac{2}{5} \) and \( \frac{3}{7} \)
are not. This is not well-defined. We want a definition of addition \underline{not dependent on our representatives chosen}.

Now, \( \displaystyle \frac{a}{b} + \frac{c}{d} = \frac{0}{1} \). This is well-defined but not helpful.

\begin{definition}[Addition in \( \mathbb{Q} \)]
    \[ \frac{a}{b} + \frac{c}{d} = \frac{ad+bc}{bd} \]
\end{definition}

If this well-defined?
\begin{proof}[Proof.]
    Assume \( (a,b) \sim (a',b') \) and \( (c,d) \sim (c',d') \). Show \( (ad+bc, bd) \sim (a'd'+b'c', b'd') \).
    \( (a,b) \sim (a',b') \implies ab' = ba' \) \\
    \( (c,d) \sim (c',d') \implies cd' = dc' \)
    \setlength{\belowdisplayskip}{0pt}
    \begin{flalign*}
        b'd'(ad+bc) &= b'd'ad + b'd'bc \\
        &= (d'd)(ab') + (b'b)(cd') \\
        &= (d'd)(ba') + (b'b)(dc') \\
        &= (bd)(a'd') + (bd)(c'b') \\
        &= bd(a'd'+c'b') & \\
    \end{flalign*}
    \( \implies (ad+bc, bd) \sim (a'd'+b'c', b'd') \) \checkmark
\end{proof}

\begin{definition}[Multiplication in \( \mathbb{Q} \)]
    \[ \frac{a}{b}\cdot\frac{c}{d} = \frac{ac}{bd} \]
\end{definition}

If this well-defined?
\begin{proof}[Proof.]
    Assume \( (a,b) \sim (a',b') \) and \( (c,d) \sim (c',d') \). Show \( (ac, bd) \sim (a'c', b'd') \). \\
    \( (a,b) \sim (a',b') \implies ab' = ba' \) \\
    \( (c,d) \sim (c',d') \implies cd' = dc' \) \\
    \setlength{\abovedisplayskip}{0pt}
    \setlength{\belowdisplayskip}{0pt}
    \begin{flalign*}
        acb'd' &= (ab')(cd') \\
        &= (ba')(dc') \\
        &= (a'c')(bd) & \\
    \end{flalign*}
    \( \implies (ac, bd) \sim (a'c', b'd') \) \checkmark
\end{proof}

In what way does \( \mathbb{Q} \) extend \( \mathbb{Z} \)?

The correspondence is \( \displaystyle \frac{n}{1} \longleftrightarrow n \).
Addition and multiplication is the same in \( \mathbb{Q} \) as in \( \mathbb{Z} \).

\begin{note}
    We can define subtraction by adding the negative of a number (multiply by \( -1 \)).
\end{note}

%%%%%%%%%%%%%%%%%%%%%%%%%%%%%%%%%%%%%%%%%%%%%%%%%%%%%%%%%%%%%%%%%%%%%%%%%%%%%%%%%%%%%%%%%%%%%%%%%%%%%%%%%%%%%%%

\subsection{Order}

\begin{definition}[Order]
    An \textbf{order} on a set \( S \) is a relation \( < \) satisfying:
    \begin{enumerate}
        \item (Trichotomy) If \( x,y \in S \), exactly one is true: \( x<y, \, x=y, \, y<x \).
        \item (Transitivity) If \( x,y,z \in S, \, x<y, \, \text{ and } y<z, x<z\).
    \end{enumerate}
\end{definition}

\begin{example}[]
    In \( \mathbb{Z} \), say \( m<n \) if \( n-m \) is positive, i.e. in \( \mathbb{N} \).
\end{example}

\begin{example}[]
    In \( \mathbb{Z} \times \mathbb{Z} \), say \( (a,b)<(c,d) \) if \( a<c \) or ( \( a=c \) or \( b<d \) ).
    This is called the dictionary order.
\end{example}

\begin{example}
    In \( \mathbb{Q} \), say \( \displaystyle \frac{m}{n} \) is positive if \( mn > 0 \).
    This is well-defined.
    \setlength{\abovedisplayskip}{0pt}
    \setlength{\belowdisplayskip}{0pt}
    \begin{proof}
        Assume \( (m,n) \sim (p,q) \) and \( mn > 0 \). Show \( pq > 0 \). \\
        Suppose, to the contrary, \( pq < 0 \).
        \begin{flalign*}
            (m,n) \sim (p,q) &\implies mq = np \\
            &\implies (mq)^2 = mqnp &
        \end{flalign*}
        By assumption, \( mnpq < 0 \), a contradiction since \( mn > 0 \).
        Thus, \( pq > 0 \).
    \end{proof}

    So \( \displaystyle \frac{a}{b} < \frac{c}{d} \) if \( \displaystyle \frac{c}{d} + \frac{-a}{b} \) is positive.
\end{example}

Write \( y>x \) for \( x<y \) and \( x <= y \) for \( x<y \) or \( x=y \).

\begin{theorem}[]
    \( x^2 = 2 \) has no solution in \( \mathbb{Q} \).
\end{theorem}

\begin{proof}[Proof (by contradiction)]
    Assume \( x^2 \) has a solution in \( \mathbb{Q} \),
    i.e. \( \displaystyle x = \frac{p}{q} \) where \( \displaystyle p,q \in \mathbb{Z} \). \\
    Also assume \( p,q \) are in ``lowest terms," i.e. they have no common factors.
    (We can do this using elements in the equivalence classes of \( \mathbb{Q} \).) \\
    So \( \displaystyle \left( \frac{p}{q}\right)^2 = 2 \), hence \( p^2 = 2 q^2 \). \\
    Then \( p^2 \) is even (divisible by 2). \\
    Then \( p \) is even. (If \( p \) was odd, \( p^2 \) would be odd.) \\
    So \( p = 2m \) for some \( m \in \mathbb{Z} \), hence \( p^2=4m^2=2q^2 \). \\
    Then \( 2m^2 = q^2 \). \\
    Then \( q^2 \) is even, hence \( q \) is even. \\
    This contradicts the fact that \( p,q \) are in ``lowest terms."
    So, \( x^2 = 2 \) must have no solution in \( \mathbb{Q} \).
\end{proof}

%%%%%%%%%%%%%%%%%%%%%%%%%%%%%%%%%%%%%%%%%%%%%%%%%%%%%%%%%%%%%%%%%%%%%%%%%%%%%%%%%%%%%%%%%%%%%%%%%%%%%%%%%%%%%%%

\subsection{Fields}

\begin{definition}[Field]
    A \textbf{field} is a set \( F \) with two operations \( +, \times \) satisfying axioms:
    \begin{axioms}{A}
        \item F is closed under \( + \). (Adding two things in the set gives you something in the set.)
        \item \( + \) is commutative.
        \item \( + \) is associative.
        \item \( F \) has an additive identity, call it 0.
        \item Every element has an additive inverse.
    \end{axioms}
    \begin{axioms}{M}
        \item F is closed under \( \times \).
        \item \( \times \) is commutative.
        \item \( \times \) is associative.
        \item \( F \) has an additive identity, call it \( 1 \).
        \item Every element except \( 0 \) has an additive inverse.
    \end{axioms}
    \begin{axioms}{D}
        \item \( \times \) distributes over \( + \).
    \end{axioms}
\end{definition}

\begin{example}[]
    In \( \mathbb{Q} \), the \( 0 \) element is \( \displaystyle \left[ \frac{0}{1} \right]\)
    and the \( 1 \) element is \( \displaystyle \left[ \frac{1}{1} \right] \).


\end{example}

\begin{definition}[Ordered Field]
    An \textbf{ordered field} is a field with an order s.t. order is preserved by field operations.
    \begin{enumerate}
        \item If \( y<z \), then \( x+y <x+z \).
        \item If \( y<z \) and \( x>0 \), then \( xy<xz \).
    \end{enumerate}
\end{definition}

\begin{note}
    \( \mathbb{Z} \) is a ring not a field. There are no multiplicative inverses. \\
    \( \mathbb{Q} \) is an ordered field!
\end{note}

\end{document}